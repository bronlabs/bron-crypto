
\inputAlgorithm{ot/extension/softspoken}{softspoken.tex}

\paragraph{SoftspokenOT Extension.} We implement the $\fn{\binom{2}{1}\mhyphen ROT}$ Extension from SoftspokenOT~\cite{softspokenot}. The protocol is depicted in Protocol \ref{alg:rote}. The protocol comprises three rounds, and requires a one-time setup of $\kappa$ batched Base OTs where the Sender/Receiver roles are reversed w.r.t. the ROT extension. To realize this protocol, we the parameters of \cite[Figure 12]{softspokenot} to $p=q_{softspoken}=2$, $k=1$, $\mathcal{C}=Rep(\mathbb{F}_2^n)$. The notation for our protocol follows closely that of an analogous description of a $\fn{\binom{2}{1}\mhyphen ROT}$ from ~\cite[Figure 10]{keller2015actively}. We perform several modifications with respect to ~\cite{keller2015actively}:


\begin{enumerate}
    \item Following the definitions from SoftspokenOT~\cite{softspokenot}, and diverging from Figure 10 of \cite{keller2015actively}, we apply the Fiat-Shamir transformation~\cite{fiatShamirTransform} to convert the interactive coin-flipping exchange for the challenge generation into a non-interactive computation based on the currently exchanged transcript, maintaining UC security of the transformed protocol as corroborated by the authors~\cite[Private communication]{DKLSCommunication}. As a clarifying note, the consistency check of the protocol can be framed as an interactive proof \fn{IP} without a commitment phase (therefore it is not a sigma protocol) and with a single challenge generation. Consequently, it suffices to query the random oracle only once in the Random Oracle Model (ROM), an therefore the \textit{salt} that Fiat-Shamir transformation requires~\cite[Section 14]{snargsBook2024} is immediately achieved by appending the \text{sid} to the transcript at the beginning of the protocol.
    \item Following the authors' recommendations, we set the statistical security parameter $\sigma = \kappa$ to align the statistical security with the computational security given that we are using the Fiat-Shamir transformation~\cite[Private communication]{DKLSCommunication}.
    \item We generalize the behavior of the protocol to run $\xi$ instances of the protocol in parallel, each with messages of $\ell$ blocks of $\kappa$ bits each, to accommodate the requirements of the $\fnref{RVOLE}$ protocol.
\end{enumerate}

Overall, the protocol relies on a one-time setup of $\kappa$ Base OTs whose results are used as persistent seeds for all executions of the protocol. A first \textit{expansion} phase (step 3 of $\mathcal{R}$.\fnref[rote]{Round1} and step 1 of $\mathcal{S}$.\fnref[rote]{Round2}) extends these seeds and establishes the correlation with the receiver's choice bits $x$ (step 4 of $\mathcal{R}$.\fnref[rote]{Round1} and step 2 of $\mathcal{S}$.\fnref[rote]{Round2}). To induce the same choice inside all bits of each message, the receiver repeats his choice bits $\ell$ times ( $\vect{x_{\text{rep}}}$ in step 1 of $\mathcal{R}$.\fnref[rote]{Round1}) and then concatenates $\sigma$ additional random choice bits ($\vect{x_{\sigma}}$) to be consumed as part of the consistency check. A subsequent \textit{consistency check} phase, made non-interactive via the Fiat-Shamir transformation, provides security against a malicious receiver\footnote{The protocol is secure against a malicious sender by design, as the random messages $(\vect{m_0}, \vect{m_1})$ arise from an expansion of the baseOT seeds  $(\vect{k_0}, \vect{k_1})$} (steps 5-6 of $\mathcal{R}$.\fnref[rote]{Round1} and steps 3-4 of $\mathcal{S}$.\fnref[rote]{Round2}). The protocol concludes with a \textit{re-randomization} phase, where both parties break the correlation of their output messages with the Base OT choice bits while maintaining the correlation with the \fnref{ROTe} receiver's choice bits (steps 7-8 of $\mathcal{R}$.\fnref[rote]{Round1} and steps 5-6 of $\mathcal{S}$.\fnref[rote]{Round2}).