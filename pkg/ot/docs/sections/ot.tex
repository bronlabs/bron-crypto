% ---------------------------------------------------------------------------- %
\subsubsection{Oblivious Transfer}\label{sec:mpc_ot}
% ---------------------------------------------------------------------------- %
An Oblivious Transfer (OT) functionality consists of a two-party interaction where a sender transfers several messages to a receiver, and the receiver chooses a subset of these messages and receives them. Both parties remain oblivious to the other party's actions: the sender is unaware of messages the receiver chose, and the receiver does not learn the content of the messages he didn't choose.

There are multiple flavors of OTs according to the number of messages, choices and the relationship established among the inputs \& outputs. For our OT functionalities, we stick to senders sending two messages and receivers receiving one message. These functionalities are:
\begin{itemize}
    \item \textit{Standard}: In the standard $\fn{\binom{2}{1}\mhyphen OT}$ described in Functionality \ref{alg:standard_ot}, the sender inputs $\kappa$-bit messages, and the receiver inputs a choice bit. The receiver then receives the message corresponding to his choice bit.
    \inputFunctionality{ot}{ot.tex}

    \item \textit{Random}: In the random $\fn{\binom{2}{1}\mhyphen ROT}$ described in Functionality \ref{alg:rand_ot}, the sender doesn't input any messages. Instead, his messages are randomly sampled by the functionality. The relationship between the inputs and outputs is the same as in the standard OT.    
    \inputFunctionality{ot}{rot.tex}

    \item \textit{Correlated}: In the correlated $\fn{\binom{2}{1}\mhyphen COT}$ described in Functionality \ref{alg:corr_ot}, a random OT is modified to enforce a mathematical relationship (over a group operation for group $\G$) between the inputs and outputs of the protocol, such that $z_a + z_B = a \cdot x$.
    \inputFunctionality{ot}{cot.tex}
\end{itemize}

These functionalities are tightly related. In fact, both the standard OT and the Correlated OTs are commonly constructed from a Random OT~\cite{mansy2019endemic}. To achieve standard OT, the sender uses the ROT output messages to one-time-pad encrypt (XOR) his two input messages, and the receiver can only decrypt one of the two encrypted messages. Similarly, a Correlated OT sender maps the ROT output messages into a group $\G$ and uses them to create and send a correlating mask, and the receiver can apply this mask to his ROT output to establish the desired correlation. Based on the techniques from \cite{mansy2019endemic}, we specify the $\fn{ROT}\rightarrowtail \fn{OT}$ composition in Protocol \ref{alg:rot2ot}, and the $\fn{ROT}\rightarrowtail  \fn{COT}$ composition in Protocol \ref{alg:rot2cot}. Consequently, the remaining protocols in this section focus only on the ROT functionality. All protocols are generalized to run a batch of $\xi$ OTs in parallel (with $\xi$ choice bits), and all ROT/OT/COT messages are composed of $\ell$ elements, where each element is either a $\kappa$-bit string (ROT/OT) or an element of a group $\G$ (COT).


\inputAlgorithm{ot}{rot2ot.tex}

\inputAlgorithm{ot}{rot2cot.tex}


OT protocols can realize these functionalities with or without relying on pre-processing generated in a a setup phase, yielding two classes of OT protocols:
\begin{itemize}
    \item \textit{Base OTs} do not rely on a setup phase, yet they require public-key cryptography and thus incur in higher computation costs, higher per-bit communication costs and more rounds to achieve. As such, they are used mostly as a building block in the one-time setup of OT extensions.
    \item \textit{OT Extensions} rely on the prior execution of a few Base OTs in a setup phase, and \textit{extend} each Base OT by seeding a PRNG with the Base OT outputs and generating a pseudo-random sequence that holds the OT relationship. Semi-honest security is achieved cheaply without communication, whereas malicious security requires some communication. 
\end{itemize}

We implement three OT protocols in our suite: two Base $\fn{\binom{2}{1}\mhyphen ROT}$ named \textit{Batched Simplest OT}~\cite[Figure 3]{mcquoid2021batching} and \textit{Verifiable Simplest OT}~\cite[Protocol 7]{doerner2018secure}, and a $\fn{\binom{2}{1}\mhyphen ROT}$ Extension named SoftspokenOT~\cite{softspokenot}. 

\inputSection{ot/base/vsot}{vsot.tex}

\inputSection{ot/base/bbot}{bbot.tex}

\inputSection{ot/extension/softspoken}{softspoken.tex}