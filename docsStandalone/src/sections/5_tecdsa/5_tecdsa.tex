% ○ Why do we need two protocols? Mention “crypto agility”, the situation with KOS, and the clients.  [Alireza]
% § What happens to the DKG?
% ○ Mention DKLs has fewer assumptions, but not as battle tested tools. Lindell has more assumptions, but battle tested.[Alireza]
% ○ We pick DKLs because t-of-n and performance. [Alireza]
% Talk about recovery process and recovery id (modify DKLS output to include recoveryId) [Alireza]
%------------------------------------------------------------------------------%
\newpage
\section{Threshold ECDSA}\label{sec:tecdsa}
%------------------------------------------------------------------------------%
 
We support two threshold ECDSA protocols: \cite{DKLS23} \& \cite{lindell17}. 


\paragraph*{Why two protocols?}
MPC can provide strong security guarantees, if done correctly. Yet, this is an area of active research and it is important to pick the right primitives, with few standard assumptions and battle-tested pieces. But bugs and security vulnerabilities still occur and some bugs require deep analysis that takes time. Depending on where the bug is, it is not always necessary to suspend all company operations to correctly assess the situation. 

Consider the case of well-known \cite{keller2015actively} protocol which was commonly deployed in threshold ECDSA protocols based on OT extension. \cite{softspokenot} invalidated the security proof of KOS15 and demonstrated a concrete attack if the security parameter is a multiple of 20. Less than a month later, \cite{diamond2022security} claimed full proof of security circumventing \cite{softspokenot} and later clarified to be providing asymptotic security for large security parameters. Although Roy did not demonstrate a concrete attack for the actual common security parameters, both \cite{keller2015actively} authors and \cite{diamond2022security} eventually recommended using a different OT extension protocol. For real-world scenarios, this translates into months of work involving R\&D, development, audit and deployment.

It is therefore necessary to have an alternative solution to which one can switch, while the main protocol goes under re-evaluations. We set \cite{DKLS23} as our main t-ECDSA protocol, and \cite{lindell17} as a backup protocol. 

% a. DKLs23: [Alberto]
% § Differences from the paper:
%     □ Typo
%     □ Zero share sampling not maintaining an index (confirm if that’s the c ase)
%     □ DKG
%     □ Base OT
% § Some points on performance
% § Mention it’ll still be UC even if fiat shamir (author explanation)
% Mention Zero share sampling for 2 party issue (author explanation)
%------------------------------------------------------------------------------%
\subsection{DKLs23}\label{sec:dkls23}
%------------------------------------------------------------------------------%

As our principal threshold ECDSA protocol, we select the 3-round $t$-out-of-$n$ signing protocol of \cite{DKLS23} realizing the standard ECDSA signature scheme (Scheme \ref{alg:ecdsa}) with UC security against $t-1$ static corruptions. The full description can be found in Protocol 3.6 of \cite{DKLS23}. We are largely faithful to the original paper, save for the following changes:
\begin{itemize}
    \item We fix a typo in the second consistency check (Step 8 of Protocol 3.6 in \cite{DKLS23}), writing $\pk_j$ instead of $\lambda_j \cdot \party{j}$.
    \item For the $\idealfn{Zero}$ functionality, instead of concatenating the seeds with an index, we salt the shared seeds with the session id. This session id is derived from the \fn{AgreeOnRandom} primitive (\autoref{sec:mpc_agreeonrandom}) per signing session.
    \item $R_i$ is included in the partial signature, for the aggregator to compute the sum and the recovery ID independently while normalizing the signature.
    \item For DKG we resort to GJKR05~\cite{gennaroDKG} instead of their pick based on the DKG of DKLs19~\cite{DKLs19}.
    \item For the discrete log proofs we resort to the randomized Fischlin transformation (Figure 9 of \cite{kondiShelatTransform}, described in Scheme \ref{alg:fischlin}) to achieve non interactivity, replacing the common Fiat-Shamir transform \cite{fiatShamirTransform} that would not be UC secure in this context.
    \item As the paper recommends and separately confirmed by the authors, the OT extension required by the $\idealfn{RVOLE}$ functionality (and instantiated with SoftSpokenOT~\cite{softspokenot} in our case) is made non-interactive with Fiat-Shamir while retaining UC security\footnote{This is achieved, among other reasons, thanks to the existence of fresh randomness in the transcript, tied to the generation of a fresh session ID per run of the protocol.}.
\end{itemize}


\subsubsection{Random VOLE} 

DKLs23's signing protocol builds upon a $\idealfn{RVOLE}$ functionality, which we detail below in the honest case\footnote{The extra details for the malicious case are detailed in Functionality 3.5 of \cite{DKLS23}}. We incorporate an additional optimization suggested in Section 5.1 of \cite{DKLS23}.

\begin{functionality}
    \caption{$\quad{\idealfndef{RVOLE}}_{q, \ell}(\vect{a}) \longrightarrow (b, \vect{c}, \vect{d})$}
    \label{alg:f_rvole}
    \begin{algorithmic}[1]
        \begin{alginfo}
            Random Vector Oblivious Linear Evaluation interacting with two parties Alice $\party{A}$ and Bob $\party{B}$.
        \end{alginfo}
        \Players{ A sender $\party{A}$, and a receiver $\party{B}$.}
        \Require $\party{A} \leftarrow \vect{a} \in \Z_q$, a vector of multiplicative shares.
        \Ensure 
            $\party{A} \leftarrow \vect{c} \in \Z_q^\ell$, a vector of additive shares.\\
            $\party{B} \leftarrow (b \in \Z_{q}, \vect{d} \in \Z^\ell_q)$, a choice bit and the additive shares.
        \vspace{-6pt}
        \AlgRoundZero{$\idealfn{RVOLE}$}{Sampling}{}{$b, \vect{c}$} Sample and send $\sample{b}{\Z_q}$ to $\party{B}$, sample and send $\sample{\vect{c}}{\Z^\ell_q}$ to $\party{A}$.
        \vspace{-18pt}
        \AlgRoundZero{$\idealfn{RVOLE}$}{Multiplication}{$\vect{a}$}{$\vect{d}$} Receive $\vect{a}$ from $\party{A}$, send $\vect{d}$ to $\party{B}$ s.t. $\vecti{c}{i}+\vecti{d}{i} = \vecti{a}{i} \cdot b \;\;\forall i \in \range{\ell}$.
    \end{algorithmic}
\end{functionality}

To realize this functionality, we follow the instructions from \cite{DKLS23} to implement a forced-reuse variant the $\pi^\ell_{2PMul}$ protocol from \cite[Protocol 1]{DKLs19} where $b$ is fixed for all the elements in the batch of size $\ell$ by reusing Bob's OT instances. Protocol \ref{alg:rvole} details the implementation of this variant. Trivially, by providing random input values, this protocol becomes a randomized multiplication protocol. We fix a typo\footnote{Corroborated by the authors~\cite{DKLSCommunication}} from the original paper in our step 2.7, correctly writing $\vecti{\theta}{i,k}$ instead of $\vecti{\theta}{k,k}$

\subsubsection{Signing} 

We detail our instantiation of \cite[Protocol 3.6]{DKLS23} in Protocol \ref{alg:dkls23_sign}. The \fn{Init} for \cite{DKLS23} is composed of a standard DKG (picked from Section \ref{sec:key_generation}), the setup for $\idealfn{Zero}$ functionality (generating the random pairwise seeds), and the setup for the $\idealfn{RVOLE}$ functionality (e.g., running the base OTs for OTe).

\subsubsection{Random VOLE} 

DKLs23's signing protocol builds upon a $\idealfn{RVOLE}$ functionality, which we detail below in the honest case\footnote{The extra details for the malicious case are detailed in Functionality 3.5 of \cite{DKLS23}}. We incorporate an additional optimization suggested in Section 5.1 of \cite{DKLS23}.

\inputFunctionality{threshold/mult/dkls23}{rvole.tex}

To realize this functionality, we follow the instructions from \cite{DKLS23} to implement a forced-reuse variant the $\pi^\ell_{2PMul}$ protocol from \cite[Protocol 1]{DKLs19} where $b$ is fixed for all the elements in the batch of size $\ell$ by reusing Bob's OT instances. Protocol \ref{alg:rvole} details the implementation of this variant. Trivially, by providing random input values, this protocol becomes a randomized multiplication protocol. We fix a typo\footnote{Corroborated by the authors~\cite{DKLSCommunication}} from the original paper in our step 2.7, correctly writing $\vecti{\theta}{i,k}$ instead of $\vecti{\theta}{k,k}$

\inputAlgorithm{threshold/mult/dkls23}{rvole.tex}


\begin{protocol}[H]
    \caption{$\quad\fn{DKLs23.Sign}$}
    \label{alg:dkls23_sign}
    \begin{algorithmic}[1]
        \begin{alginfo}
            $t$-out-of-$n$ threshold signing protocol from \cite{DKLS23}, realizing the standard \fnref{ECDSA} functionality with UC security for a group $\G(q,\g{G})$.  The protocol builds on $\fnref[gennarodkg]{DKG}$, $\fnref{Przs}$, $\fnref{RVOLE}_{2,q}$, a $\fnref[dlc]{Commitment}$ scheme and a hash function $\fn{H}$.
        \end{alginfo}  
        \Players \begin{itemize}[leftmargin=0pt,label=, itemsep=-2pt]
            \item Key share holders: $\{\party{i}\}_{i \in \range{n}}$ holding $\{x_i\}_{i \in \range{n}}$ and public key $\g{Q}$
            \item \hspace{28pt}Quorum of signers: $\{\party{i}\}_{i \in \set{S}}$ for $\set{S} \ins \subrange{t}{n}$ and $\set{S}^* \equals \set{S} \setminuss \{i\}$
        \end{itemize}
        \Require A session identifier $sid$, and a message $\vect{m}$
        \Ensure A partial signature $\sigma_i$ per $\party{i}$. A signature $\sigma$ after aggregation

        \vspace{-12pt}
        \AlgRoundZero{$\party{i}$}{\fndef[dkls23]{Init}}{}{$(x_i, \g{Q}, \zeta_i)$}
            \State Run $\assign{(\g{Q}, x_i)}{\fnref[gennarodkg]{DKG}}$ to obtain a public and a private key share
            \State Run $\fnref{Przs}.\fnref[przs]{Setup1}()$, $\fnref{Przs}.\fnref[przs]{Setup2}()$ and $\fnref{Przs}.\fnref[przs]{Setup3}()$ to setup zero sharing
                \State Run $\fnref{RVOLE}.\fnref[rvole]{Setup}()$  as Alice with $\party{k}$ as Bob $\forall k \ins \range{n} \setminuss \{i\}$
                \State Run $\fnref{RVOLE}.\fnref[rvole]{Setup}()$  as Bob with $\party{k}$ as Alice $\forall k \ins \range{n} \setminuss \{i\}$

        \vspace{-12pt}
        \setalglineno{1}
        \AlgRound{$\party{i}$}{\fndef[dkls23]{Round1}}{}{$(R_i, \{c'_{ij}, \gamma_{ij}\}_{j \in \set{S}^*})$}
            \State Sample $\sample{\phi_i}{\Z_q}$ as an inversion mask and $\sample{r_i}{\Z_q}$ as an instance key
            \State $\assign{R_i}{r_i \cdot \g{G}}$ as the public instance key
            \For{$j \in \set{S}^*$}
                \State Run $\assign{(c'_{ij},w_{ij})}{\fnref[blc]{Commit}(i \concat j \concat sid \concat R_i)}$
                \State Run $\assign{(\gamma_{ij}, b_{ij})}{\fnref{RVOLE}.\fnref[rvole]{Round1}()}$ as Bob
                \State $\sendTo{\party{j}}{c'_{ij}, \gamma_{ij}}$
            \EndFor
            \State $\idealfnref{Broadcast}(R_i)$

        \vspace{-12pt}
        \setalglineno{1}
        \AlgRound{$\party{i}$}{\fndef[dkls23]{Round2}}{$\{R_j, c'_{ji}, \gamma_{ji}\}_{j \in \set{S}^*}$}{$(\{\vect{\mu}^{rnd2}_{ij}, \Gamma^u_{ij}$, $\Gamma^v_{ij}$, $b_{ij}, w_{ij}\}_{j \in \set{S}^*}, R_i, P_i)$}
            \State Run $\assign{\zeta_i}{\fnref{Przs}.\fnref[przs]{Sample}()}$ to get a zero share
            \State $\assign{a_i}{\fnref{ShamirToAdditive}(i, \set{S}, x_i)}$
            \State $\assign{sk_i}{a_i + \zeta_i}$ and $\assign{P_i}{sk_i \cdot \g{G}}$ as refreshed instance key shares
            \For{$j \in \set{S}^*$}
                \State Run $\assign{(\vect{\mu}^{rnd2}, \vect{c}\!\equiv\!\{c^u,c^v\})_{ij}}{\fnref{RVOLE}.\fnref[rvole]{Round2}(\gamma_{ij}, \vect{a}\equals\{r_i, sk_i\})}$ as Alice
                \State $\assign{\Gamma^u_{ij}}{c^u_{ij} \cdot \g{G}}$ and $\assign{\Gamma^v_{ij}}{c^v_{ij} \cdot \g{G}}$
                \State $\assign{\psi_{ij}}{\phi_i - b_{i,j}}$
                \State $\sendTo{\party{j}}{\vect{\mu}^{rnd2}_{ij}, \Gamma^u_{ij}$, $\Gamma^v_{ij}$, $b_{ij}, w_{ij}, R_i}$ 
            \EndFor
            \State $\idealfn{Broadcast}(P_i)$

        \vspace{-12pt}
        \setalglineno{1}
        \AlgRound{$\party{i}$}{\fndef[dkls23]{Round3}}{$\vect{m}, \{\vect{\tilde{a}}_{ji}, \vect{\eta}_{ji}, \mu_{ji}, \Gamma^u_{ji}$, $\Gamma^v_{ji}$, $b_{ji}, w_{ji}, P_j\}_{j \in \set{S}^*}$}{$\sigma_i$}
        \For{$j \in \set{S}^*$}
            \State Run $\fnref[blc]{Open}(j || i || sid || R_j, c'_{ji}, w_{ji})$, $\ABORT$ if it fails
            \State Run $\assign{(\vect{d}\equiv\{d^u_{ij},d^v_{ij}\})}{\fnref{RVOLE}.\fnref[rvole]{Round3}(\vect{\mu}^{rnd2}_{ij}\equals\{\vect{\tilde{a}}, \vect{\eta}, \mu\})}$ as Bob
            \State Check if $\isequal{b_{ji} \cdot R_j - \Gamma^u_{ji}}{ d^u_{ij} \cdot \g{G}}$ otherwise $\ABORT$
            \State  Check if $\isequal{b_{ji} \cdot P_i - \Gamma^v_{ji} }{d^v_{ij} \cdot \g{G}}$ otherwise $\ABORT$
        \EndFor
        \State Check if $\isequal{\sum_{j\in\set{S}} \g{P}_j}{\g{Q}}$, otherwise $\ABORT$
        \State $\assign{R}{\sum_{j\in\set{S}} R_j}$
        \State $\assign{u_i}{r_i \cdot (\phi_i + \sum_{j\in\set{S}^*} \psi_{ji}) + \sum_{j\in\set{S}^*} (c^u_{ij} + d^u_{ij})}$
        \State $\assign{v_i}{sk_i \cdot (\phi_i + \sum_{j\in\set{S}^*} \psi_{ji}) + \sum_{j\in\set{S}^*} (c^v_{ij} + d^v_{ij})}$
        \State $\assign{w_i}{\fn{SHA2}(\vect{m}) \cdot \phi_i + (R_x) \cdot v_i}$
        \State \Return $\sigma_i \equals \{u_i, w_i\}$

        \algstore{dkls23_sign}
    \end{algorithmic}
\end{protocol}
\begin{protocol}[H]
    \begin{algorithmic}[1]
        \algrestore{dkls23_sign}
        \vspace{-6pt}
        \Statex
        \AlgFnct{\fndef[dkls23]{Aggregate}}{}{$\g{Q}, \{u_j, w_j, R_j\}_{j \in \set{S}}$}{$\sigma\;$}
        \setalglineno{1}
        \State $\assign{R}{\sum \g{R}_j}$ and $\assign{r}{\g{R}_x}$
        \State $\assign{s}{\frac{\sum w_j}{\sum u_j}}$
        \State $\assign{v}{(\g{R}_y \mod 2) + 2( \isgeq{\g{R}_x}{q})}$ as the recovery identifier $\in \Z_4$
        \If{$ \isgt{(-s \mod q)}{s}$} \Commentg{(Normalize to "low $s$ form")}
            \State $\assign{s}{(-s) \mod q}$
            \State $\assign{v}{(v + 2) \mod 4}$
        \EndIf
        \State Run $\fnref{ECDSA}.\fnref[ecdsa]{Verify}(\g{Q}, \vect{m}, \sigma \equals(r,s,v))$ to check if the signature is valid
        \Statex \Return $\sigma \equals (r, s, v)$ as the signature
    \end{algorithmic}
\end{protocol}








% b. Lindell17: [Mateusz]
% § Differences from the paper:
%     □ Incomplete DKG
%     □ Changes to the proofs in DKG
%     □ Shamir Share instead of additive.
%     □ Signing differences with simplifications
% Some points on performance

\subsection{Lindell17}\label{sec:lindell17}

We also support the $2$-out-of-$n$ signing protocol of Lindell17~\cite[Protocol 3.6]{lindell17} realizing the standard ECDSA signature scheme (\ref{alg:ecdsa}) with UC security against $1$ malicious static corruption. We are largely faithful to the original paper, save for two changes:
\begin{enumerate}
    \item \textit{Decomposition of private key share.} In the Lindell17's DKG protocol, parties sample a random share $x$ in a range $[q/3,2q/3]$ to make it compatible with the ZK proofs presented in the paper. However, in our use case the $x$ is sampled during the DKG protocol (Protocol \ref{alg:gennarodkg}) in the full range $[0, q)$. To bridge this gap, we reformulate the provided $x$ as a combination of $x_1$ and $x_2$ such that $x = 3x_1 + x_2 \mod q$ and both $x_1$ and $x_2$ are in the specified range $[q/3,2q/3]$, making both of them ZKP-compatible. We resort to Algorithm \ref{alg:decomposetwothirds}, and prove correctness of this decomposition in Appendix~\ref{sec:proof_lindell17dkg}.
    
    \begin{algorithm}[H]
        \caption{$\quad\fndef{DecomposeTwoThirds}_q(x\in \Z_q) \rightarrow (x_1, x_2)$}\label{alg:decomposetwothirds}
        \begin{algorithmic}[1]
            \For{$i \in \{0,1,2\}$}
                \If{$x \in \left[\frac{3k}{18}q, \frac{3(k + 1)}{18}q \right)$} 
                    \State Sample $\sample{x_1}{\left[\frac{9 + k}{18}q, \frac{9 + (k + 1)}{18}q \right)}$
                \EndIf
            \EndFor
            \For{$i \in \{3,4,5\}$}
                \If{$x \in \left[\frac{3k}{18}q, \frac{3(k + 1)}{18}q \right)$}
                    \State Sample $\sample{x_1}{\left[\frac{3 + k}{18}q, \frac{3 + (k + 1)}{18}q \right)}$
                \EndIf
            \EndFor
            \Statex \Return $(x_1, x - 3x_1)$
        \end{algorithmic}
    \end{algorithm}
    
    As a consequence, instead of storing $c_{key} = \paillier{x}$ in the last step of the protocol, we store $c_{key} = 3 \odot \paillier{x'} \oplus \paillier{x''}$. 
    \item \textit{Shamir shares of private key}: we use Shamir secret sharing (SSS) instead of multiplicative sharing for the private key. This is a minor change that does not affect the security of the protocol, but it allows us to use the same private key sharing as that defined in the DKG protocol (see \autoref{sec:key_generation}).
\end{enumerate}

We describe the altered version of Lindell17's DKG in Protocol \ref{alg:lindell17dkg}. 
The signing protocol is  implemented to faithfully match the original, with a only minor deviation: instead of using multiplicative SS for the private key, we employ additive SS derived from Shamir SS. The protocol is described in Protocol \ref{alg:lindell17sign}.

\input{src/algorithms/threshold_signatures/lindell17_dkg.tex}   

\input{src/algorithms/threshold_signatures/lindell17_sign.tex}
