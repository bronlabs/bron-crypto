%------------------------------------------------------------------------------%
\section{Signing}\label{sec:signing}
%------------------------------------------------------------------------------%

FROST signing protocol is a two-round algorithm where in the first round, participants generate single-use nonce pairs and commit to them, and in the second round, the commitments are used to produce partial signatures using a Shamir-to-additive share conversion technique \cite{CDI05}. To perform such conversion faster, optimization is possible by precomputing the Lagrange coefficients. The storage cost is ${n \choose t} \times t$ many coefficients. We do not make any assumption on whether this optimization is present and write the pseudocode without this optimization.

Finally, Partial signatures are then aggregated by the $\mathcal{SA}$ and result in the actual signature. For more information on the $\mathcal{SA}$ role, refer to section \ref{roles}.

\subsection{Interactive}\label{sign:interactive}
The full signing protocol is interactive and is defined in Algorithm \ref{alg:sign:online}.

% \begin{breakablealgorithm}
    \caption{$\{(\sigma, m), \{\}\} \gets (t, n)$ FROST Signature of $m$}\label{alg:sign:online}
    \begin{algorithmic}[1]
        \Require $t$ Participants are present and have previously performed a key generation and agree on the inputs and outputs therein and agree who assumes the role of $\mathcal{SA}$ and $\mathcal{SR}$ and agree on hash functions $H_1, H_2 : \left \{ 0, 1 \right \}^{\star} \rightarrow \mathbb{Z}_q$ and let $S$ be the set $\alpha: t\leq \alpha \leq n$ participants present for this signing operation
        \Statex
        \Statex Each participant $P_i$ calls \hyperref[alg:sign:online:round1]{Round1}, then \hyperref[alg:sign:online:round2]{Round2}. If $i \in \mathcal{SA}$ then $P_i$ will run \hyperref[alg:sign:online:aggregate]{Aggregate} after  \hyperref[alg:sign:online:round2]{Round2}.
        \Statex $\mathcal{SR}$ should send $m$ before  \hyperref[alg:sign:online:round2]{Round2} starts.
        \Statex
        \Procedure {Round1}{} \label{alg:sign:online:round1}
            \State \textcolor{red}{$(d_i, e_i) \overset{\$}{\leftarrow} \mathbb{Z}^{\star}_q \times \mathbb{Z}^{\star}_q $ and store it in memory.} \label{noncepair}
            \State \textcolor{red}{$(D_i, E_i) \gets (d_i \cdot G, e_i \cdot G)$ and store it in memory.}
            \State \textcolor{teal}{Pass $(i, D_i, E_i)$ to \hyperref[func:broadcast]{$\mathcal{F}_{broadcast}$}.}
        \EndProcedure
        \Statex
        \Procedure {Round2}{
        
                $\quad m,$
                
                $\quad Ds \gets \left \{ D_\alpha: \alpha \in S \land \alpha \neq i \right \},$

                $\quad Es \gets \left \{ E_\alpha: \alpha \in S \land \alpha \neq i \right \}$
                
            } \label{alg:sign:online:round2}
            \Statex \textbf{Require: } the sender of $m$ is in $\mathcal{SR}$ and $0 \not \in Ds$ and $0 \not \in Es$.
            \State $digest \gets H_2(m)$
            \State $(\bm{D_\alpha}, \bm{E_\alpha}) \gets \hyperref[alg:sign:online:presig]{\textsc{ProcessNonceCommitmentOnline}}(Ds, Es)$
            \State $R \gets 0_G$
            \For{$j \in S$}
                \State $r_j \gets H_1(j, digest, \bm{D_\alpha}, \bm{E_\alpha})$ 
                \State $R_j \gets D_j + r_j \cdot E_j$
                \State $R \gets R + R_j$
            \EndFor
            \State $c \gets H_1(R, Y, digest)$
            \State Derive own's Lagrange coefficient $\lambda_i \gets \prod_{j\in S \land j \neq i} \frac{j}{j-i}$
            \State Compute partial signature element $z_i \gets d_i + e_i r_i + \lambda_i s_i c$
            \State \textcolor{red}{Securely delete $d_i$ and $e_i$ from memory.}
            \For{$j \in \mathcal{SA}$}
                \If{$j \in S$}
                    \If{$j = i$}
                        \State \textcolor{red}{Store $m, z_i, R, \bm{D_\alpha}, \bm{E_\alpha}$ in memory.}
                    \EndIf
                    \State \textcolor{teal}{Pass $(j, z_i)$ to \hyperref[func:p2psend]{$\mathcal{F}_{P2PSend}$}.}
                \Else
                    \Comment{\textcolor{gray}{$j^{th}$ signature aggregator is not one of the participants.}}
                    \State \textcolor{teal}{Pass $(z_i, \bm{D_\alpha}, \bm{E_\alpha})$ to the signature aggregator in another way.}
                \EndIf
            \EndFor
        \EndProcedure
        \Statex
        \Procedure {Aggregate}{

            $\quad \bm{zs} \gets \left \{ z_\alpha: \alpha \in S \right\},$

            $\quad m$ if $m$ not in memory, 

            $\quad D_\alpha^\prime \gets \{(j, \bm{D_\alpha}) : j \in S\}$ if $\bm{D_\alpha}$ not in memory,

            $\quad E_\alpha^\prime \gets \{(j, \bm{E_\alpha}) : j \in S\}$ if $\bm{E_\alpha}$ not in memory,

        } \label{alg:sign:online:aggregate}
            
            \Statex \textbf{Require: } the sender of $m$ is in $\mathcal{SR}$ and $D_\alpha^\prime$ has only one element and $E_\alpha^\prime$ has only one element.

            \State $digest \gets H_2(m)$

            \If{$D_\alpha^\prime$ is provided}
                \State Let $\bm{D_\alpha} \gets D_\alpha^\prime \left [ 0 \right ]$
            \EndIf
            \If{$E_\alpha^\prime$ is provided}
                \State Let $\bm{E_\alpha} \gets E_\alpha^\prime \left [ 0 \right ]$
            \EndIf

            \If{$R$ not in memory}
                \State $R \gets 0_G$
                \For{$j \in S$}
                    \State $r_j \gets H_1(j, digest, \bm{D_\alpha}, \bm{E_\alpha})$
                    \State $R_j \gets D_j + r_j \cdot E_j$
                    \State $R \gets R + R_j$
                \EndFor
            \EndIf
            
            \For{$j \in S$}
                \Comment{\textcolor{gray}{If Identifiable Abort is not needed, skip this loop.}}\label{identifiableabort}
                \State $c \gets H_1(R, Y, digest)$
                \State compute public key share $Y_j \gets \sum_{l=1}^{n} \sum_{k=0}^{t-1} jk \cdot C_{(l, k)} \mod{q}$
                \State $\lambda_j \gets \prod_{k\in S \land k \neq j} \frac{k}{k-j}$
                \If{$z_j \cdot G \neq R_j + c\lambda_j \cdot Y_j$}
                    \State \textcolor{red}{\textsc{ABORT WITH}} $P_j$ as misbehaving.
                \EndIf
            \EndFor
            \State $z = \sum \bm{zs}$
            \State $\sigma \gets (R, z)$
            \State Pass $(\sigma, Y, m)$ to \hyperref[section:eddsa]{$\mathcal{F}_{EdDSA\_Verify}$} and \textcolor{red}{\textsc{ABORT}} if it fails.
            \State \Return $(\sigma, m)$
        \EndProcedure
        \Statex
        \Function{ProcessNonceCommitmentsOnline}{$Ds, Es$} \label{alg:sign:online:presig}
            \State $\bm{D_\alpha} \gets {Ds} \cup \{D_i\}$
            \State $\bm{E_\alpha} \gets {Es} \cup \{E_i\}$
            \For{$j \in S$}
                \If{$j \neq i$}
                    \If{$D_j \not \in G^{\star} \lor E_j \not \in G^{\star}$}
                        \State \textcolor{red}{\textsc{ABORT}}
                    \EndIf
                \EndIf
            \EndFor
            \State \Return $(\bm{D_\alpha}, \bm{E_\alpha})$
        \EndFunction
    \end{algorithmic}
\end{breakablealgorithm}

\subsection{Non-Interactive with Presignature generation}\label{noninteractive}
FROST is \textit{semi non-interactive} as it accepts the message only at the final round. Therefore the first round could be processed in batches, making the protocol non-interactive. To discuss the non-interactive variant of FROST further, we will call $(d_i, e_i)$ as generated Algorithm \ref{sign:interactive} Round \ref{alg:sign:online:round1} line \ref{noncepair}, as \textsc{nonce-pair} of participant $P_i$. We will also define a \textsc{presignature} to be $((d_1 \cdot G, e_1 \cdot G), \ldots, (d_n \cdot G, e_n \cdot G))$ which is basically the commitment to the nonce-pair of all parties.

We adopt \hyperref[alg:sign:online:round1]{Round1} of Algorithm \ref{alg:sign:online}. The result, however, has two rounds because all parties need access to the entire presignatures of other parties to maintain an internal mapping from their private nonce-pairs to the corresponding presignatures - recall that we need to maintain such a mapping to compute $R$. It is therefore implied that our preprocessing phase is interactive. It is possible to have a non-interactive preprocessing phase if there exists an append-only, auditable, tamper-evident ledger accessible by all parties. We would not discuss the non-interactive variant any further, as it is not in scope for this primitive.  

Note that the nonce-pairs are private, and each participant should store them securely. The presignatures are public and should be stored in a manner accessible by $\mathcal{PC}$. We do not make any assumptions about where and how preprocessing materials are stored. The presignatures themselves do not require any form of encryption (because we cannot extract the scalar pairs from the presignatures due to the difficulty of the discrete logarithm). 

We also require participants of the non-interactive variant of FROST to record what nonce-pairs were used. It is \textbf{critical} for nonce-pair to be used only once. We do not make any assumption on how the user of this primitive prevents nonce-pair reuse as it is context-dependent.

The non-interactive variant of FROST also requires an additional signing keypair to authenticate nonce-pairs to achieve TS-SUF-4 level of unforgeability \cite{BCKMTZ22}. In our implementation, we will serialize each participant's additional Schnorr public key and consider it the participant's cohort ID (see \ref{cohortconfig}).

Finally, note that, unlike online signing, for simplicity, we require that all participants who are permitted to sign be present during presignature generation. 

%Also, it is recommended to refresh the key once all presignatures are used. For more information on the key refresh process, see section \ref{refresh}. 

% 
\begin{breakablealgorithm}
    \caption{$(((d_1, e_1), PS_1), \ldots, ((d_\tau, e_\tau), PS_\tau)) \gets PreGen(\tau)$}\label{alg:pregen}
    \begin{algorithmic}[1]
        \Require All $n$ participants are present, have previously performed a key generation, and agree on $\tau > 1$. Each participant $P_i$ has an additional Schnorr keypair $(w_i, W_i)$ where the public key $W_i$ is known by all participants.
        \Ensure $PS_1, \ldots, PS_\tau$ are persistently stored such that they are accessible by $\mathcal{PC}$, and $(d_1, e_1), \ldots, (d_\tau, e_\tau)$ are securely stored persistently, and $(d_1, e_1), \ldots, (d_\tau, e_\tau)$ are securely deleted from memory after the pregeneration ceremony has ended.
        \Statex
        \Statex Each participant $P_i$ calls \hyperref[alg:pregen:round1]{Round1}, then \hyperref[alg:pregen:round2]{Round2}. 
        \Statex
        \Procedure {Round1}{} \label{alg:pregen:round1}
            \For{$j \in [1, \tau]$}
                \State \textcolor{red}{$(d_i^{(j)}, e_i^{(j)}) \overset{\$}{\leftarrow} \mathbb{Z}^{\star}_q \times \mathbb{Z}^{\star}_q $ and securely store it in memory.}
                \State $(D_i^{(j)}, E_i^{(j)}) \gets (d_i^{(j)} \cdot G, e_i^{(j)} \cdot G)$
                \State $\sigma_{D_i^{(j)}} \gets \hyperref[alg:schnorrsign]{SchnorrSign(w_i, D_i^{(j)}, \{i\})}$
                \State $\sigma_{E_i^{(j)}} \gets \hyperref[alg:schnorrsign]{SchnorrSign(w_i, E_i^{(j)}, \{i\})}$ 
                \State \textcolor{red}{$PS_{(i, j)} \gets ((D_i^{(j)}, \sigma_{D_i^{(j)}}), (E_i^{(j)}, \sigma_{E_i^{(j)}}))$ and store it in memory.}
                % \State Records $[((D_i^{(j)}, \sigma_{D_i^{(j)}}), (E_i^{(j)}, \sigma_{E_i^{(j)}}))]$ in $\mathcal{F}_{Ledger}$
            \EndFor
        \State \textcolor{teal}{Pass $(PS_{(i, 1)}, \ldots, PS_{(i, \tau)})$ to \hyperref[func:broadcast]{$\mathcal{F}_{broadcast}$}.}
        \EndProcedure
        \Statex
        \Procedure {Round2}{
        
                $\quad (PS_{(j, 1)}, \ldots, PS_{(j, \tau)})$ for $j \in [1,n] \land j \neq i$
                
            } \label{alg:pregen:round2}
            \Statex \textbf{Require: } Not any point received is zero.
            \For{$j \in [1, \tau]$}
                \Comment{\textcolor{gray}{$PS_j$ is the $j^{th}$ presignature.}}
                \State $PS_j \gets (PS_{(1, j)}, \ldots, PS_{(n, j)})$
            \EndFor
            \State \Return $(((d_1, e_1), PS_1), \ldots, ((d_\tau, e_\tau), PS_\tau))$
        \EndProcedure
    \end{algorithmic}
\end{breakablealgorithm}

The non-interactive signing is just \hyperref[alg:sign:online:round2]{Round2} of the online signing as defined in Algorithm \ref{alg:sign:online}, with one small modification: The function call of \hyperref[alg:sign:online:presig]{\textsc{ProcessNonceCommitmentsOnline}} is changed to that of \hyperref[alg:sign:offline]{\textsc{ProcessPresig}} as defined in Algorithm \ref{alg:sign:offline}.

\begin{breakablealgorithm}
    \caption{Non-Interactive Signing}\label{alg:sign:offline}
    \begin{algorithmic}[1]
        \Require All participants agree on who assumes the role of $\mathcal{PC}$ and participants have done presignature ceremony as defined in Algorithm \ref{alg:pregen} $\tau > 1$ many times and each participant $P_i$ has an additional Schnorr keypair $(w_i, W_i)$ where the public key $W_i$ is known by all participants and everything required by the online signing as defined in Algorithm \ref{alg:sign:online}
        \Statex
        \Statex Each participant $P_i$ calls \hyperref[alg:sign:online:round2]{Round2} swapping \hyperref[alg:sign:online:presig]{\textsc{ProcessNonceCommitmentOnline}} with \hyperref[alg:sign:offline:func]{ProcessPresig}. Input messages to \hyperref[alg:sign:offline:func]{ProcessPresig} are supplied by $\mathcal{PC}$.
        \Statex If $i \in \mathcal{SA}$ then $P_i$ will run \hyperref[alg:sign:online:aggregate]{Aggregate} afterwards.
        \Statex If $\mathcal{SR}$ is not one of the participants, then $\mathcal{SR}$ should send $m$ to unblock \hyperref[alg:sign:online:round2]{Round2}.
        \Statex
        \Function {ProcessPresig}{$PS_k$ for some $1 < k \leq \tau$} \label{alg:sign:offline:func}
            \Statex \textbf{Require: } $k$ and the corresponding presignature is not used before, and the presignature matches the internal copy $P_i$.
            \State Parse $PS_k$ as $((D_1^{(k)}, \sigma_{D_1^{(k)}}), (E_1^{(k)}, \sigma_{E_1^{(k)}})), \ldots, ((D_n^{(k)}, \sigma_{D_n^{(k)}}), (E_n^{(k)}, \sigma_{E_n^{(k)}}))$
            \For{$j \in S$}
                \If{$\neg $ \hyperref[alg:schnorrverify]{SchnorrVerify($\sigma_{D_j^{(k)}}$, $D_j^{(k)}$, $\{ j \}$)}}
                    \State \textcolor{red}{\textsc{ABORT}}
                \EndIf
                \If{$\neg $ \hyperref[alg:schnorrverify]{SchnorrVerify($\sigma_{E_j^{(k)}}$, $E_j^{(k)}$, $\{ j \}$)}}
                    \State \textcolor{red}{\textsc{ABORT}}
                \EndIf
            \EndFor
            \If{$d_i^{(k)}$ not in memory $\lor$ $e_i^{(k)}$ not in memory}
                \State \textcolor{red}{\textsc{ABORT}}
            \EndIf
            \If{$d_i^{(k)} \cdot G \neq D_i^{(k)}$}
                \State \textcolor{red}{\textsc{ABORT}}
            \EndIf
            \If{$e_i^{(k)} \cdot G \neq E_i^{(k)}$}
                \State \textcolor{red}{\textsc{ABORT}}
            \EndIf
            \State $\bm{D_\alpha} \gets \left \{ D_\alpha: \alpha \in S \right \}$
            \State $\bm{E_\alpha} \gets \left \{ E_\alpha: \alpha \in S \right \}$
            % \State Records $(\bm{D_\alpha}, \bm{E_\alpha})$ to $\mathcal{F}_{Ledger}$
            \State \Return $(\bm{D_\alpha}, \bm{E_\alpha})$
        \EndFunction
    \end{algorithmic}
\end{breakablealgorithm}