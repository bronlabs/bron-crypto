% a. General: [Alireza]
% § Mention the verify function and security game etc
\section{Digital Signatures}\label{sec:signatures}


A digital signature is an asymmetric cryptographic primitive for verifying the authenticity of digital messages or documents \cite{goldwasser1983strong,goldwasser1988digital,bellare1988sign,rompel1990one,goldwasser2008lecturenotes}. A valid signature on a message gives a recipient trust in that the message was authored by a sender known to the recipient. Digital signatures are commonly used for financial transactions, software distribution, and to detect forgery or tampering in communication.

Assuming that the private key remains secret, digital signatures are designed to be inherently \textit{resistant to forging}, making it computationally infeasible to generate a valid signature on behalf of a party without knowing that party's private key. They may also provide \textit{non-repudiation}, thus signer cannot successfully claim they did not sign a message. Digitally signed messages may be anything that can be represented as a bit-string, e.g., cryptocurrency transactions, electronic mail, or messages sent as part of cryptographic protocols.

We generalize digital signature schemes in Functionality \ref{alg:f_signature}:

\begin{functionality}
    \caption{$\quad\idealfndef{Signature}$}
    \label{alg:f_signature}
    \begin{algorithmic}[1]
        \begin{alginfo}
            Signature scheme, composed of three algorithms:
            \begin{itemize}[leftmargin=12pt]
                \item $\fndef[f_signature]{KeyGen}()\!\rightarrow\! (\sk,\pk)$, generates a private \& public key pair $(\sk, \pk)$
                \item $\fndef[f_signature]{Sign}(\sk, m)\!\rightarrow\! \sigma$ yields signature $\sigma$ for message $m$ \& private key $\sk$
                \item $\fndef[f_signature]{Verify}(\pk, m, \sigma)\!\rightarrow\! b \ins \{0,1\}$ verifies signature $\sigma$ for message $m$ and public key $\pk$, either accepting ($b\equals1$) or rejecting ($b\equals0$) it\footnotemark.
            \end{itemize}
        \end{alginfo}
    \end{algorithmic}
\end{functionality} 

\footnotetext{Equivalently, it returns \textit{valid} ($b\equals1$) for a valid signature and $\ABORT$ otherwise.}

The security of digital signature schemes is filled with nuances. In general, the security of a digital signature scheme is defined by a security game between a challenger and an adversary. Based on the capabilities granted to the adversary, we distinguish three basic attacks (more details in \cite{goldwasser1983strong}):
\begin{itemize}
    \item \textit{Key-Only Attack (KOA)}: the adversary knows only the public key of the signer, thus it can verify signatures of messages given to him.
    \item \textit{Known Signature Attack (KSA)}: the adversary is given the public key of the signer and message/signature pairs from a legal signer.
    \item \textit{Chosen Message Attack (CMA)}: The adversary is also given access to a signing oracle, a black-box that signs any message of the adversary's choice. Out of the three, this is the most powerful adversary and is considered the only realistic notion for many real-world scenarios.
\end{itemize}

\noindent The adversary's goal in the security game may be:
\begin{itemize}
    \item \textit{Existential Forgery (EF)}: succeed in forging the signature of one message, not necessarily of his choice.
    \item \textit{Selective Forgery (SF)}: succeed in forging the signature of some message of his choice.
    \item \textit{Universal Forgery (UF)}: succeeds in forging the signature of any message.
    \item \textit{Total Break}: succeed in computing the signer's secret key.
\end{itemize}

The signature scheme is then considered secure if the adversary's probability of winning the game for a given goal and attack is negligible. We refer the avid reader to \cite{goldwasser2008lecturenotes} for more formal definitions.

There exist many different digital signature schemes, each with their own performance characteristics. In this section, we focus our attention to the most common signature schemes used in blockchain protocols: ECDSA~\cite{ecdsa_rfc6979}, Schnorr / EdDSA~\cite{eddsa_rfc8032}, and BLS~\cite{irtf_bls_signature_05}. These three schemes are proven secure under the \textit{UF-CMA} security definition\footnote{Some signature schemes such as Schnorr / EdDSA go even beyond, as they constitute \textit{zero knowledge proofs of knowledge} of the secret key.}, the strongest combination out of the classical security notions presented above. All these schemes are based on elliptic curve cryptography, for specifically chosen curves.

\input{src/sections/3_signatures/3_ecdsa}
\input{src/sections/3_signatures/3_schnorr}
% d. BLS:   [Alireza]
% § Mention target curve for each algorithm
% § Signing modes
% Briefly mention use-cases
% ---------------------------------------------------------------------------- %
\subsection{BLS}\label{sec:bls}
% ---------------------------------------------------------------------------- %


The Boneh-Lynn-Shacham (BLS) signature scheme~\cite{boneh2001short} is a deterministic, non-malleable, and efficient signature scheme grounded on pairing-based cryptography, resorting to a type III bilinear pairing $\fn{e}\!: \G_1 \times \G_2 \rightarrow \G_T$ for signature verification. Subject of an RFC draft~\cite{irtf_bls_signature_05}, the BLS signature scheme is provably secure (the scheme is \textit{existentially unforgeable} under \textit{adaptive chosen-message attacks}) in the random oracle model assuming hardness of a variant of the computational Diffie-Hellman problem. $\G_1$ and $\G_2$ are elliptic curve groups with the same characteristic $q$ defined over fields of order $p$ and $p^2$ respectively, and $\G_T$ is a target group over a field of order $p^{12}$.


\begin{scheme}[ht]
    \caption{$\quad\fndef{BLS}$}
    \label{alg:bls}
    \begin{algorithmic}[1]
        \begin{alginfo}
            The pairing-based signing scheme from \cite{boneh2001short} over curve BLS 123 81 \cite{irtf_bls_signature_05}, instantiated with short keys \textit{wlog} and with all rogue key prevention schemes. It uses hash functions $\fn{H}_{\G_1}$ and $\fn{H}_{\G_2}$ over $\G_1$ and $\G_2$ respectively. $\mathbf{BLS123\mhyphen81}.\fn{Verify}(m, \sigma)$ is a pairing-based verification function returning $\isequal{(\fn{H}_{\G_1}(m) \times \g{Y}^{-1}) \cdot (\g{G}_1 \times \sigma)}{1}$.
        \end{alginfo}
        \Require {A unique session identifier $sid$, a message $\vect{m}$ to be signed, a public key $\g{Y}_i$, and a private key $x_i$ for each signer $\party{i} \;\forall i \in \range{t}$.}

        \Ensure {A partial signature $\sigma_i$ for each player $\party{i} \;\forall i \in \range{t}$ after round 1, and a signature $\sigma$ after aggregation, }

        \vspace{-12pt}
        \setalglineno{1}
        \AlgFnctZero{\fndef[bls]{Sign}}{i}{$\vect{m}$}{$\sigma_i$}
            \State If $\fn{MessageAug}$, $\assign{\vect{m}}{\g{Y}_i \concat  \vect{m} }$.
            \State If $\fn{PoP}$,  $\assign{\pi_i}{\fn x_i  \times \fn{H}_{\G_2}(\g{Y}_i) }$.
            
            \State $\sigma_i \gets x_i \times \fn{H}_{\G_2}(\vect{m})$
            \State \Return $\sigma_i$ as the partial signature, attaching $\pi_i$ to it if $\fn{PoP}$.
      
        \vspace{-12pt}
        \setalglineno{1}
        \AlgFnct{\fndef[bls]{Verify}}{}{$\{\sigma_i\}_{i \in \range{t}}$}{\textit{valid}}
            \For{$i \in \range{t}$}
                \State If $\fn{Basic}$, ensure all $\vecti{m}{i}$ are unique. $\ABORT$ otherwise.
                \State If $\fn{PoP}$, check if $\fn{\mathbf{BLS123\mhyphen81}.Verify}(\g{Y}_i, \pi_i)$ is \textit{valid}. $\ABORT$ otherwise.
                \State If $\fn{MessageAug}$, $\assign{\vect{m}}{\g{Y}_i \concat  \vect{m} }$.
                \State Check if $\fn{\mathbf{BLS123\mhyphen81}.Verify}(\g{Y}_i, \sigma_i)$ is \textit{valid}. $\ABORT$ otherwise.
            \EndFor
            \Statex \Return \textit{valid}.

    %TODO: Batch aggregate verify

    \end{algorithmic}
\end{scheme}


Based on the choice of what group to use for what scheme element, BLS supports two modes of operation:
\begin{itemize}
    \item \textbf{Short public keys, long signatures}: Signatures are longer and slower to create, verify, and aggregate but public keys are small and fast to aggregate. Used when signing and verification operations not computed as often or for minimizing storage or bandwidth for public keys. $\G_1$ is used as the key group, and $\G_2$ as the signature group. 
    \item \textbf{Short signatures, long public keys}: Signatures are short and fast to create, verify, and aggregate but public keys are bigger and slower to aggregate. Used when signing and verification operations are computed often or for minimizing storage or bandwidth for signatures. $\G_2$ is used as the key group, and $\G_1$ as the signature group.
\end{itemize}

The BLS signature can be composed into a multi-signature scheme with batch verification. To do so, an additional mechanism must be introduced to prevent rogue public-key attacks (i.e., a malicious party can carefully craft a signature that negates the contributions of other parties and reveals a secret key). The BLS signature scheme supports three rogue key prevention schemes: 

\begin{enumerate}
    \item \textbf{Basic}: Requiring all messages to be unique.
    \item \textbf{Message Augmentation}: Prepends the public key of the signer to the message thereby making each message unique.
    \item \textbf{Proof of Possession (PoP)}: Every signature is accompanied by a signature of the public key acts as proof of possession of its secret key.
\end{enumerate}

We diverge slightly from \cite{irtf_bls_signature_05} by employing an optimization trick on the final exponentiation\footnote{From \url{https://hackmd.io/benjaminion/bls12-381\#Final-exponentiation}}. Crucially, we also include the message hashing as part of the verification process in order to avoid some attacks. %TODO: add reference