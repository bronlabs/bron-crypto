%------------------------------------------------------------------------------%
\subsubsection{Sigma Protocols and Non-Interactivity}\label{sec:proof_sigma}
%------------------------------------------------------------------------------%
In this context, a Sigma Protocol $\bm{\Sigma}$ is a four-move interactive PoK protocol run between a prover $\mathcal{P}$ and a verifier $\mathcal{V}$ with three distinct communication rounds:
\begin{enumerate}
    \item \fndef[sigma]{Commitment}: $\mathcal{P}$ gets $\assign{A}{\fn{Commit}(w)}$ of statement $w$ and sends it to $\mathcal{V}$.
    \item \fndef[sigma]{Challenge}: $\mathcal{V}$ computes and sends a challenge $e$ to $\mathcal{P}$.
    \item \fndef[sigma]{Response}: $\mathcal{P}$ computes and sends a response $z$ to $\mathcal{V}$.
    \item \fndef[sigma]{Verify}: $\mathcal{V}$ checks if the response is \textit{valid}, rejecting the proof otherwise.
\end{enumerate}

$\bm{\Sigma}$-protocols require heavy interaction, often employing more rounds of communication than the protocols in which they are used. To transform any interactive Sigma Protocol into a non-interactive proof, we develop a \textbf{compiler} that applies one of two well-studied transformations:


\begin{scheme}[ht]
    \caption{$\quad$\fn{Fiat\mhyphen Shamir}(\fndef{FS})} \label{alg:fiatshamir}
    \begin{algorithmic}[1]
    \begin{alginfo}
        A transformation from a public-coin Interactive-Proof (IP) $\fn{IP}_\text{R}$ into a Non-Interactive proof system following \cite{fiatShamirTransform}, parametrized by a hash function $\fn{H}$.
    \end{alginfo} 
    
    \Players {A prover $\mathcal{P}$ and a  verifier $\mathcal{V}$.}

    \vspace{-8pt}
    \AlgRound{$\mathcal{P}$}{\fndef[fs]{Prove}}{$x, w$}{$\pi$}
        \State Run all steps of the prover in $\assign{\pi}{\fn{IP}_\text{R}.\fn{Prove}(x, w)}$, replacing\footnotemark the verifier's challenges with $\assign{e_i}{\fn{H}(x \concat m_0  \dots\concat m_i)}$ where $m_i$ is $\mathcal{P}$ message at step $i$.
        \Statex \Return $\pi$
        
        
    \vspace{-8pt}
    \setalglineno{1}
    \AlgRound{$\mathcal{V}$}{\fndef[fs]{Verify}}{$x , \pi$}{\textit{valid}}        
        \State Run all steps of the verifier in $\fn{IP}_\text{R}.\fn{Verify}(x, \pi)$, replacing the verifier's challenges with $\assign{e_i}{\fn{H}(x \concat m_0  \dots\concat m_i)}$ where $m_i$ is $\mathcal{P}$ message at step $i$.
        
    \end{algorithmic}
\end{scheme}
\footnotetext{In practice we employ a transcript \fnref{T} (Section \ref{sec:mpc_transcript}), replacing message concatenation with $\fnref{T}.\fnref[t]{Append}$ and the final hashing with the hash-chaining of $\fnref{T}.\fnref[t]{Extract}$.}


\begin{itemize}
    \item \textbf{Fiat-Shamir Heuristic}~\cite{fiatShamirTransform}: replaces the challenge $e$ with a hash of the commitment $A$ and the transcript of the previous rounds. This popular non-interactive transformation is used throughout our suite of protocols, such as the Schnorr signing scheme described in Scheme \ref{alg:schnorr} or the consistency check as part of the Oblivious Transfer Extension from Protocol \ref{alg:rote}. Despite its simplicity, the Fiat-Shamir heuristic is not universally composable (UC) secure, as it requires forking to extract the simulator's view. This limitation is addressed by the Fischlin transform, which we describe next. A more recent take on the Fiat-Shamir heuristic can be consulted in \cite{snargsBook2024}. 
    

    \item \textbf{Fischlin Transform}~\cite{fischlinTransform}: replaces the challenge $e$ with a set of hashes of the commitment $A$ with random values $r$ sampled by the prover $\mathcal{P}$. Contrary to Fiat-Shamir, the Fischlin transform is straight-line extractable without forking, allowing it to be proven UC secure. Despite this, a known limitation of the Fischlin transform is that it applies to a limited class of Sigma Protocols with a “quasi-unique response” property, which doesn't necessarily permit standard compositions for Sigma protocols (e.g. one proof OR another proof). A randomized variant of Fischlin proposed by Kondi-Shelat~\cite[Section 6.4]{kondiShelatTransform} removes this limitation, and thus we select it for our compiler.

    
\begin{scheme}[ht]
    \caption{$\quad$\fndef{Fischlin}} \label{alg:fischlin}
    \begin{algorithmic}[1]
    \begin{alginfo}
        A transformation from an Interactive Sigma Protocol $\bm{\Sigma}_\text{R}$ into a Non-Interactive (NI) proof system following \cite{fischlinTransform}, parametrized by a hash function\footnotemark $\fn{H}_\ell\!: \{0,1\}^* \mapsto \Z_2^\ell$ with digest length of $\ell$ bits (we set $\ell\equals8$).
    \end{alginfo} 
    
    \vspace{-8pt}
    \AlgFnct{\fndef[fischlin]{Prove}}{}{$\text{R}, w$}{$\pi$}
        \For {$i \in \range{\lambda/\ell}$}
            \State $\assign{(a_i, \text{state}_i)}{\bm{\Sigma}_\text{R}.\fnref[sigma]{Commitment}(w,\text{R})}$ commits witness $w$ for relation $\text{R}$
        \EndFor
        % \State Set $\assign{\vect{A}}{\{\g{A}_1,\dots,\g{A}_{\lambda/\ell}\}}$
        \For {$i \in \range{\lambda/\ell}$}
            \State Sample a challenge $\sample{e_i}{\{0,1\}^{\ell \log_2(\lambda)}}$
            \State $\assign{z_i}{\bm{\Sigma}_\text{R}.\fnref[sigma]{Response}(\text{state}_i, e_i)}$ as the response to that challenge.
            \State Check if $\isequal{\fn{H}_\ell(\{a_i\}_{\forall i} \concat i \concat e_i \concat z_i)}{\vect{0}_{\Z_2^\ell}}$, otherwise go back to step 6
        \EndFor
        \Statex \Return $(\pi \equals \{a_i,e_i, z_i\}_{\forall i})$
        
        
    \vspace{-8pt}
    \setalglineno{1}
    \AlgFnct{\fndef[fischlin]{Verify}}{}{$\text{R}, \pi\equals\{a_i, e_i, z_i\}_{i \in \range{\lambda/\ell}}$}{\textit{valid}}        
        \For {$i \in \range{\lambda/\ell}$}
            \State Check if $\isequal{\fn{H}_\ell(\{a_i\}_{\forall i} \concat i \concat e_i \concat z_i)}{\vect{0}_{\Z_2^\ell}}$, otherwise $\ABORT$
            \State Check if $\bm{\Sigma}_\text{R}.\fnref[sigma]{Verify}(x, z_i, e_i)$, otherwise $\ABORT$
        \EndFor
        \Statex \Return \textit{valid}
        
    \end{algorithmic}
\end{scheme}

\footnotetext{Typically instantiated via a standard hash function with truncated output, e.g., SHA-256.}

\end{itemize}

We apply our compiler to all the proofs in our suite, allowing us to use them in a non-interactive manner in our protocols. As a byproduct, we enhance our compiler to provide composition of proofs:

\begin{itemize}[leftmargin=10pt]
    \item[] $ \text{AND}(\bm{\Sigma}_1, \bm{\Sigma}_2)$: Straightforwardly achieved by running both proofs in parallel.
    \item[] $\text{OR}(\bm{\Sigma}_1, \bm{\Sigma}_2)$: Following the composition technique from \cite{damgaard2002sigma,ciampi2016improved}, it requires simulating the proof that is not used by the prover $\mathcal{P}$ while running the proof that is used in a standard fashion.
\end{itemize}

This composition forces the challenge generation in our proofs to be defined over generic bit-strings in order to generalize over different kinds of challenge spaces (i.e., $\{0,1\}^n$ or $\mathbb{Z}_q$). Additionally, all proofs must provide a \textit{simulator} that can generate a valid transcript without knowing the secret $a$ of $\mathcal{P}$.