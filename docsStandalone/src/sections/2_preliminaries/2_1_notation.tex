%------------------------------------------------------------------------------%
\subsection{Notation}\label{sec:notation}
%------------------------------------------------------------------------------%

\begin{table}[ht!]
    \begin{center}
    \caption{Notation, Symbols and Operators}\label{tab:notation}%
    \begin{tabular}{@{}ll@{}}
    \toprule
    Symbol & Description\\
    \midrule
    $a$ & An integer value $\ins \Z$, equivalent to a bit-string $\ins \{0,1\}^*$ \\
    $|a|$ & Bit-length of bit-string $a$. \\
    $\vect{a}$ & A vector of values $\{\vecti{a}{1},\vecti{a}{2}\dots\}$\\
    $\vecti{a}{i}$ & $i$th element of vector $\vect{a}$\\
    $\set{S}$ & A set with elements $\{s_1, s_2, \dots \}$\\
    $\range{n}$ & Set of integers $\{1, 2, \dots, n\}$\\
    $|\set{S}|$ & Number of elements in set $\set{S}$ (its cardinality)\\
    $s_i$ & $i$th element of a set $\set{S}$, with $1\!\leq\!i\!\leq\!|\set{S}|$ \\
    $\subrange{k}{n}$ & Set of all $k$-subsets of $\range{n}$, that is, $\subrange{k}{n} \triangleq \{\set{X}\!:\, \set{X}\! \subseteq\! \range{n} \land |\set{X}| \equals k \}$\\
    $\g{a}\equals(\g{a}_x, \g{a}_y)$ & A group element (e.g., EC point), with coordinates $\g{a}_x, \g{a}_y$\\ 
    $\poly{p}$ & A polynomial of a certain finite degree $d$\\
    $\polyx{p}{x}$ & Polynomial evaluated on point $x$\\
    $\coeff{p}{i}$ & $i$th coefficient of polynomial $\poly{p}$, s.t. $\polyx{p}{x} = \sum_{i=0}^d  \coeff{p}{i}x^i$ \\
    $\pk, \sk$ & Public and private Paillier keys \\
    $\paillier{a}$ & Paillier ciphertext, encryption of value $a$ with $\pk$\\
    $\str{tag}$ & An ascii-encoded string \\
    % $\unif{S}$ & Uniformly random distribution in the set $S$ \\
    $\assign{a}{1}$ & Assignment operator. Set the value of $a$ to $1$ \\
    $\sample{a}{\set{S}}$ & Sampling operator. Sample value $a$ uniformly from set $\set{S}$ \\
    $\appendT{a}$ & Operator to appends $a$ to the transcript \\
    $\assignAppendT{a}{1}$ & Assign value $1$ to $a$ and append $a$ to the transcript \\
    $\isequal{a}{1}$ & Equality test operator. Returns $1$ if true, $0$ otherwise \\
    $a \!\mod q$ & Modulo operator \\
    $a\concat q$ & Concatenation operator (elements, sets, bit-strings\dots) \\
    $\fn{Func}_{a}(x)$ & Function parametrized by $a$ with input $x$\\
    $\fn{gcd}(x, y)$ & Greatest common divisor of $x$ and $y$ \\
    $\fn{lcm}(x, y)$ & Least common multiple of $x$ and $y$ \\
    $\fn{quot}(x)$ &  Quotient function $\lfloor \frac{x - 1}{n} \rfloor$ \\
    $\fn{sgn}(x)$ &  Sign function, $1$ if $x>0$, $-1$ if $x<0$, $0$ if $x=0$\\
    $\party{1}, \party{2}, \dots$ & Computing parties in MPC protocol \\
    $\mathcal{S}, \mathcal{R}, \dots$ & Parties with a specific role (e.g., \textit{Sender}, \textit{Receiver}, \dots) \\
    $\party{k}.\fn{Func}(x)$ & Party $k$ runs a certain function $\fn{Func}$ on input $x$. \\
    $\idealfn[name]{}$ & An ideal functionality \\
    $\kappa, \lambda$ & Computational security parameters (typ. 128-256 bits) \\
    $\sigma, \lambda_s$ & The statistical security parameters (typ. 80-128 bits) \\
    $\G(q, \g{G})$ & Cyclic group of prime order $q$ and generator $\g{G}$ \\
    $\set{E}(\G, q, \g{G}, \g{I})$ & Elliptic curve with group $\G(q, \g{G})$ and identity element $\g{I}$ \\
   
    \bottomrule
    \end{tabular}
    \end{center}
\end{table}

We use plain low-case letters (e.g., $a, k$) for scalars, bold letters to denote vectors (e.g., $\bm{x}$, $\bm{y}$), upper-case non-italic letters for sets (e.g., $\set{S} \equals \{1,2,3\}$) and upper-case italic letters (e.g., $\g{A}$) to denote points in elliptic curves. $\vecti{x}{i}$ denotes the $i$th element of vector $\vect{x}$. We write a polynomial $\poly{p}$ of degree $d$ as $\polyx{p}{x} = \sum_{i=0}^{d-1}  \coeff{p}{i}x^i$, where $\coeff{p}{i}$ is the $i$th coefficient of $\poly{p}$. We use $\assign{q}{4}$ to set a local variable $q$ to 4, $\isequal{a}{b}$ to check whether $a$ is equal to $b$, and $a \equals b$ to denote equivalence between $a$ and $b$. We note $\unif{\set{S}}$ as the uniform random distribution in a set $\set{S}$, and write $\sample{r}{S}$ to indicate sampling from $\unif{\set{S}}$ and assigning the sample to $r$. We use $\pk, \sk$ to denote public and private keys (e.g., Paillier, signing).  We denote $\kappa$ or $\lambda$ as the computational security parameter (128-256 bits) for all our primitives and protocols, and $\sigma$ or $\lambda_s$ as the statistical security parameter.


In the context of MPC protocols, $\party{1},\party{2},\dots$ denote the computing parties. We generalize behavior common to a set of $k$ parties by resorting to $\party{i}$ for an index $i \in \{1,2,\dots,k\}$, and reserve the index $j \in \{1,2,\dots,k\}\!\setminus\!\{i\}$ for behavior common to all other parties $\party{j}$ given a certain party $\party{i}$. Certain parties fulfilling a specific roles are denoted with that same font (e.g., $\SA$ for signature aggregator, $\mathcal{S}$ for sender). We employ sans-serif fonts for functions and protocol rounds (e.g., $\fn{Round1}$ for the first round of a protocol, $\fn{gcd}(x,y)$ for the greatest common divisor of $x$ and $y$), and denote an ideal functionality as $\idealfn[name]{}$.
For convenience, we summarize our notation choices in \autoref{tab:notation}.
