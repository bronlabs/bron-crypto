
%------------------------------------------------------------------------------%
%----------------------------------MATH ENV------------------------------------%
%------------------------------------------------------------------------------%
% Entities
\newcommand{\party}[1]{\ensuremath{\mathcal{P}_{#1}}}   % Party
\newcommand{\SA}{\ensuremath{\mathcal{S\!A}}}           % Signature Aggregator
\newcommand{\PC}{\ensuremath{\mathcal{PC}}}             % Pre-signature Composer
\newcommand{\SR}{\ensuremath{\mathcal{SR}}}             % Signature Requester
\newcommand{\sender}{\ensuremath{\mathcal{S}}}          % Sender
\newcommand{\receiver}{\ensuremath{\mathcal{R}}}        % Receiver
\newcommand{\adversary}{\ensuremath{\mathcal{A}}}       % Adversary

% Brackets of different kinds
\makeatletter
\newsavebox{\@brx}
\newcommand{\llangle}[1][]{\savebox{\@brx}{\(\m@th{#1\langle}\)}
  \mathopen{\copy\@brx\mkern2mu\kern-0.9\wd\@brx\usebox{\@brx}}}  %<<
\newcommand{\rrangle}[1][]{\savebox{\@brx}{\(\m@th{#1\rangle}\)}
  \mathclose{\copy\@brx\mkern2mu\kern-0.9\wd\@brx\usebox{\@brx}}} %>>
\makeatother
\newcommand{\angled}[1]{\left \langle {#1} \right \rangle} % <x>
\newcommand{\dangled}[1]{\llangle {#1} \rrangle}           % <<x>>
\newcommand{\bracked}[1]{\left[ {#1} \right]}              % [x]
\newcommand{\dbracked}[1]{ \llbracket {#1} \rrbracket}     % [[x]]

% Functions
\newcommand{\fn}[1]{\ensuremath{\mathsf{#1}}}                        % Function #1
\newcommand{\fndef}[2][]{{\fn{{#2}}\label{func:#1#2}}}               % Function #2 with label
\newcommand{\fnref}[2][]{\fn{\hyperrefblk{func:#1#2}{#2}}}           % Function #2 with reference
% \newcommand{\idealfn}[1]{\ensuremath{\mathcal{F}_{#1}}}              % Ideal Function #1
\newcommand{\idealfn}[2][]{\ensuremath{\mathcal{F}_{#1}^{#2}}}              % Ideal Function #1
\newcommand{\idealfndef}[2][]{{\idealfn{{#2}}\label{idealfunc:#1#2}}}     % Ideal Function #2 with label
\newcommand{\idealfnref}[2][]{\idealfn{\hyperrefblk{idealfunc:#1#2}{#2}}} % Ideal Function #2 with reference

\newcommand{\sendTo}[2]{\ensuremath{\fnref{Send}(#2)\rightarrow{#1}}}
\newcommand{\send}[3]{\ensuremath{{#1}.\fnref{Send}(#3)\rightarrow{#2}}}

% Letters
\newcommand{\Z}{\mathbb{Z}}
\newcommand{\R}{\mathbb{R}}
\newcommand{\F}{\mathbb{F}}
\newcommand{\G}{\mathbb{G}}
\mathchardef\mhyphen="2D

% Notation
\newcommand{\range}[1]{\ensuremath{\bm{[}{#1}\bm{]}}}                 % range 
\newcommand{\subrange}[2]{\ensuremath{\range{#2}^{#1}}}     % subset of range 
\newcommand{\vect}[1]{\ensuremath{\bm{#1}}}                           % vector #1
\newcommand{\vecti}[2]{\ensuremath{{#1}_{({#2})}}}                    % #2th element of vector #1
\newcommand{\set}[1]{\ensuremath{\mathrm{\uppercase{#1}}}}            % mathematical set #1
\newcommand{\g}[1]{\ensuremath{\mathnormal{\uppercase{#1}}}}          % curve point
% \newcommand{\asshare}[2]{\angled{#1}_{#2}}     
\newcommand{\asshare}[2]{\ensuremath{{#1}_{#2}}}                      % secret shared value by party #2
\newcommand{\pk}{\mathit{pk}}                                         % public key
\newcommand{\sk}{\mathit{sk}}                                         % private key
\newcommand{\paillier}[1]{\dbracked{#1}}                         % Paillier encrypted value
\newcommand{\distrs}[2]{\ensuremath{\mathcal{#1}_{#2}}}               % Random Distribution in the set #2
\newcommand{\distr}[1]{\ensuremath{\mathcal{#1}}}                     % Random Distribution
\newcommand{\unif}[1]{\set{U}_{#1}}                                   % Uniform random distribution in the set #1
\newcommand{\sample}[2]{\ensuremath{{#1}\oset{\supertiny{ \$}}{\leftarrow}{#2}}}% Sample value #1 from distribution #2
\newcommand{\sampleAppendT}[2]{\ensuremath{{#1}\uset{\centerdot}{\oset{\supertiny{ \$}}{\leftarrow}}{#2}}}% Assign value #2 to variable #1, and append variable #1 to transcript
\newcommand{\assignAppendT}[2]{\ensuremath{{#1}\uset{\centerdot}{\leftarrow}{#2}}}% Assign value #2 to variable #1, and append variable #1 to transcript
\newcommand{\appendT}[1]{\ensuremath{({#1})_{\centerdot}}}
\newcommand{\checkif}[3]{\ensuremath{{#1}\oset{\supertiny{ ?}}{#2}{#3}}}  % Check condition
\newcommand{\isequal}[2]{\checkif{#1}{=}{#2}}                         % Check equality
\newcommand{\isgeq}[2]{\checkif{#1}{\geq}{#2}}                        % Check greater or equal
\newcommand{\isgt}[2]{\checkif{#1}{<}{#2}}                            % Check greater
\newcommand{\poly}[1]{\ensuremath{\bm{\mathrm{#1}}}}                  % Polynomial
\newcommand{\polyx}[2]{\ensuremath{\poly{#1}({#2})}}           % Polynomial evaluated on point #2
\newcommand{\coeff}[2]{\ensuremath{\poly{#1}_{#2}}}            % Polynomial coefficient #2
\newcommand{\assign}[2]{\ensuremath{{#1}\leftarrow{#2}}}              % Assign value #2 to variable #1
\newcommand{\concat}{\ensuremath{\;||\;}}                             % Concatenation operator
\newcommand{\equals}{\ensuremath{\!=\!}}                              % Equivalence operator
\newcommand{\ins}{\ensuremath{\!\in\!}}                               % In operator
\newcommand{\inss}{\ensuremath{\!\!\in\!\!}}                          % In operator
\newcommand{\by}{\ensuremath{\!\times\!}}                             % Dimension multiplication (e.g., 3 by 2 matrix)
\newcommand{\str}[1]{\scalebox{0.75}{\ensuremath{``\mathtt{#1}"}}}      % String
\newcommand{\setminuss}{\ensuremath{\!\setminus\!}}                   % Set minus