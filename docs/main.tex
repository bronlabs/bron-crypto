\documentclass{article}

%------------------------------HEADERS & VERSION-------------------------------%
    \usepackage[utf8]{inputenc} 
\usepackage[section]{algorithm}% Float environment for algorithms
\usepackage{algorithmicx}       % Syntax for algorithms
\usepackage[noEnd=true,indLines=false,beginComment=,rightComments=true]{algpseudocodex} % Customized algorithmicx
\usepackage{lineno}             % line numbers
\usepackage{bm}                 % bold math
\usepackage[section]{placeins}
\usepackage{xcolor}
\usepackage{comment}
\usepackage{amsmath}
\usepackage{import}             % For \subimport
\usepackage{amssymb}
\usepackage{enumitem}           % [leftmargin=12pt] in itemize
\usepackage{graphicx}           % scalebox
\usepackage{tcolorbox}          % Colored boxes for ideal functionalities
\usepackage{booktabs}           % For \toprule, \midrule and \bottomrule
\usepackage[only,llbracket,rrbracket]{stmaryrd} % For \llbracket and \rrbracket

\usepackage{amsthm}             % Theorems
\newtheorem{theorem}{Theorem}
\newtheorem{case}{Case}

\usepackage{amsfonts}           % Short minus
\DeclareMathSymbol{\shortminus}{\mathbin}{AMSa}{"39}

\usepackage[hypertexnames=false]{hyperref}           % Colour links
\hypersetup{
  colorlinks=true,
  linkcolor=blue, %Colour of internal links
}

\newcommand{\hyperrefblk}[2]{\hyperref[#1]{\color{black}{#2}}} % Black hyperlinks


%------------------------------------------------------------------------------%
%-------------------------------FILE LOCATION----------------------------------%
%------------------------------------------------------------------------------%

\newcommand{\codepath}{../pkg}
\newcommand{\inputAlgorithm}[2]{\inputfrom{\codepath/#1/docs/algorithms}{#2}}
\newcommand{\inputFunctionality}[2]{\inputfrom{\codepath/#1/docs/functionalities}{#2}}
\newcommand{\inputSection}[2]{\inputfrom{\codepath/#1/docs/sections}{#2}}


%------------------------------------------------------------------------------%
%-------------------------------GENERAL SETUP----------------------------------%
%------------------------------------------------------------------------------%
% Autoref with capitalized names
\renewcommand{\figureautorefname}{Figure}
\renewcommand{\tableautorefname}{Table}
\renewcommand{\partautorefname}{Part}
\renewcommand{\appendixautorefname}{Appendix}
\renewcommand{\chapterautorefname}{Chapter}
\renewcommand{\sectionautorefname}{Section}
\renewcommand{\subsectionautorefname}{Section}
\renewcommand{\subsubsectionautorefname}{Section}


%------------------------------------------------------------------------------%
%--------------------------------ALGORITHM ENV---------------------------------%
%------------------------------------------------------------------------------%
% ALGORITHM INPUTS, OUTPUTS, PLAYERS
\renewcommand{\algorithmicrequire}{\textbf{Inputs:}}
\renewcommand{\algorithmicprocedure}{}
\renewcommand{\algorithmicensure}{\textbf{Outputs:}}
\algnewcommand{\Players}{\item[\textbf{Players:}]}
\algnewcommand{\Params}{\item[\textbf{Parameters:}]}

% DESCRIPTIONS and COMMENTS
\algnewcommand{\Note}{\item[\textbf{Note:}]}
\algnewcommand{\Descr}[1]{{\setlength{\itemindent}{-\leftmargin}\item[]{#1}}}
\algnewcommand{\Descrx}[1]{{\setlength{\itemindent}{-\labelwidth}\item[]\textit{#1}}}
\algnewcommand{\Commentg}[1]{\Comment{\textcolor{gray}{#1}}}

% PROTOCOL
%  with "Players", "Input" and "Output"
\newcounter{protocol}
\makeatletter
\newenvironment{protocol}[1][htb]{%
    \let\c@algorithm\c@protocol
    \renewcommand{\ALG@name}{Protocol}
    \begin{algorithm}[#1]%
    }{
    \end{algorithm}
}
\makeatother

\newcounter{scheme}
\makeatletter
\newenvironment{scheme}[1][htb]{%
    \let\c@algorithm\c@scheme
    \renewcommand{\ALG@name}{Scheme}
    \begin{algorithm}[#1]%
    }{
    \end{algorithm}
}
\makeatother

% FUNCTIONALITY
%  with "Players", "Input" and "Output"
\newcounter{functionality}
\makeatletter
\newenvironment{functionality}[1][htb]{%
    \let\c@algorithm\c@functionality 
    \renewcommand{\ALG@name}{Functionality}
    \begin{tcolorbox}[left=3pt,right=3pt]
    \vspace{-15pt}
        \begin{algorithm}[H]
        }{
        \end{algorithm}
    \end{tcolorbox}
    }
\makeatother
  
% Set Algorithmic line number
\newcommand{\setalglineno}[1]{
  \setcounter{ALG@line}{\numexpr#1-1}}

% Gray comment in protocols / algorithms
\newcommand{\ABORT}{{\textcolor{red}{\textsc{ABORT}}}}
\newcommand{\grayhfill}{\color{lightgray}\hrulefill}

% Change margin
\newenvironment{changemargin}[2]{%
  \begin{list}{}{%
    \setlength{\topsep}{0pt}%
    \setlength{\leftmargin}{#1}%
    \setlength{\rightmargin}{#2}%
    \setlength{\listparindent}{0pt}%
    \setlength{\labelsep}{0pt}%
    \setlength{\itemindent}{0pt}%
    \setlength{\parsep}{\parskip}%
  }%
  \item[]}{\end{list}}

% Algorithm info (paragraph at the beginning of the algorithm)
\newenvironment{alginfo}{
    \item[]
    \begin{changemargin}{-16pt}{0pt} 
} {\end{changemargin}\vspace{4pt} }

% Round syntax style
\algnewcommand{\RoundFmt}[2]{\textup{#1}$.$\textbf{\textsf{#2}}}
% First round in protocol
\algnewcommand{\AlgRoundZero}[4]{
    \Statex % Always use one empty state to mark the beginning of the protocol.
    \begin{changemargin}{-32pt}{0pt} 
        \Descrx {\RoundFmt{#1}{#2}}({#3})$\dashrightarrow${#4}
    \end{changemargin}}
% Next protocol rounds
\algnewcommand{\AlgRound}[4]{
    \begin{changemargin}{0pt}{0pt} 
        \Descrx {\RoundFmt{#1}{#2}}({#3})$\dashrightarrow${#4}
    \end{changemargin}}
\algnewcommand{\AlgRoundInput}{\par$\quad$} % \item for inputs of each round.
\algnewcommand{\AlgSeparator}{{\color{lightgray}\hrulefill}\vspace{4pt}} % Round separator.

% Function syntax style
\algnewcommand{\FunctFmt}[2]{\ensuremath{\bm{\textsf{#1}}_{#2}}}
% First Function in scheme
\algnewcommand{\AlgFnctZero}[4]{
    \Statex % Always use one empty state to mark the beginning of the protocol.
    \begin{changemargin}{-32pt}{0pt} 
        \Descrx {\FunctFmt{#1}{#2}}({#3})$\dashrightarrow${#4}
    \end{changemargin}}
% Next functions
\algnewcommand{\AlgFnct}[4]{
    \begin{changemargin}{0pt}{0pt} 
        \Descrx {\FunctFmt{#1}{#2}}({#3})$\dashrightarrow${#4}
    \end{changemargin}}

%------------------------------------------------------------------------------%
%---------------------------------CUSTOM ENVS----------------------------------%
%------------------------------------------------------------------------------%
% INDENTATION
% Environment to indent right content based on first parameter, then indent the first line of the content by the second parameter.
% e.g., indenting the first line by 20pt and the rest by 30pt:
%      \begin{indent}{30pt}{-10pt} 
%           My Paragraph.
%      \end{indent}
\newenvironment{indentator}[2]
  { \setlength{\leftskip}{#1}
    \bgroup
    \setlength{\parindent}{#2}
  }{\egroup
    \setlength{\leftskip}{0pt}
  }


%------------------------------------------------------------------------------%
%----------------------------------AUXILIARY-----------------------------------%
%------------------------------------------------------------------------------%

% TODOs
\newcommand{\todo}[1]{\textcolor{orange}{[Todo: #1]}}

% Super-tiny font
\newcommand{\supertiny}[1]{\scalebox{.5}{#1}}

% \oset macro for tight overset, from https://tex.stackexchange.com/a/194805/288479
\makeatletter
\newcommand{\oset}[3][0ex]{%
  \mathrel{\mathop{#3}\limits^{
    \vbox to#1{\kern-2\ex@
    \hbox{$\scriptstyle#2$}\vss}}}}
\makeatother

% \uset macro for tight underset, from https://tex.stackexchange.com/a/194805/288479
\makeatletter
\newcommand{\uset}[3][0ex]{%
  \mathrel{\mathop{#3}\limits_{
    \vbox to#1{\kern-10\ex@
    \hbox{$\scriptstyle#2$}\vss}}}}
\makeatother
    
%------------------------------------------------------------------------------%
%----------------------------------MATH ENV------------------------------------%
%------------------------------------------------------------------------------%
% Entities
\newcommand{\party}[1]{\ensuremath{\mathcal{P}_{#1}}}   % Party
\newcommand{\SA}{\ensuremath{\mathcal{S\!A}}}           % Signature Aggregator
\newcommand{\PC}{\ensuremath{\mathcal{PC}}}             % Pre-signature Composer
\newcommand{\SR}{\ensuremath{\mathcal{SR}}}             % Signature Requester
\newcommand{\sender}{\ensuremath{\mathcal{S}}}          % Sender
\newcommand{\receiver}{\ensuremath{\mathcal{R}}}        % Receiver
\newcommand{\adversary}{\ensuremath{\mathcal{A}}}       % Adversary

% Brackets of different kinds
\makeatletter
\newsavebox{\@brx}
\newcommand{\llangle}[1][]{\savebox{\@brx}{\(\m@th{#1\langle}\)}
  \mathopen{\copy\@brx\mkern2mu\kern-0.9\wd\@brx\usebox{\@brx}}}  %<<
\newcommand{\rrangle}[1][]{\savebox{\@brx}{\(\m@th{#1\rangle}\)}
  \mathclose{\copy\@brx\mkern2mu\kern-0.9\wd\@brx\usebox{\@brx}}} %>>
\makeatother
\newcommand{\angled}[1]{\left \langle {#1} \right \rangle} % <x>
\newcommand{\dangled}[1]{\llangle {#1} \rrangle}           % <<x>>
\newcommand{\bracked}[1]{\left[ {#1} \right]}              % [x]
\newcommand{\dbracked}[1]{ \llbracket {#1} \rrbracket}     % [[x]]

% Functions
\newcommand{\fn}[1]{\ensuremath{\mathsf{#1}}}                        % Function #1
\newcommand{\fndef}[2][]{{\fn{{#2}}\label{func:#1#2}}}               % Function #2 with label
\newcommand{\fnref}[2][]{\fn{\hyperrefblk{func:#1#2}{#2}}}           % Function #2 with reference
% \newcommand{\idealfn}[1]{\ensuremath{\mathcal{F}_{#1}}}              % Ideal Function #1
\newcommand{\idealfn}[2][]{\ensuremath{\mathcal{F}_{#1}^{#2}}}              % Ideal Function #1
\newcommand{\idealfndef}[2][]{{\idealfn{{#2}}\label{idealfunc:#1#2}}}     % Ideal Function #2 with label
\newcommand{\idealfnref}[2][]{\idealfn{\hyperrefblk{idealfunc:#1#2}{#2}}} % Ideal Function #2 with reference

\newcommand{\sendTo}[2]{\ensuremath{\fnref{Send}(#2)\rightarrow{#1}}}
\newcommand{\send}[3]{\ensuremath{{#1}.\fnref{Send}(#3)\rightarrow{#2}}}

% Letters
\newcommand{\Z}{\mathbb{Z}}
\newcommand{\R}{\mathbb{R}}
\newcommand{\F}{\mathbb{F}}
\newcommand{\G}{\mathbb{G}}
\mathchardef\mhyphen="2D

% Notation
\newcommand{\range}[1]{\ensuremath{\bm{[}{#1}\bm{]}}}                 % range 
\newcommand{\subrange}[2]{\ensuremath{\range{#2}^{#1}}}     % subset of range 
\newcommand{\vect}[1]{\ensuremath{\bm{#1}}}                           % vector #1
\newcommand{\vecti}[2]{\ensuremath{{#1}_{({#2})}}}                    % #2th element of vector #1
\newcommand{\set}[1]{\ensuremath{\mathrm{\uppercase{#1}}}}            % mathematical set #1
\newcommand{\g}[1]{\ensuremath{\mathnormal{\uppercase{#1}}}}          % curve point
% \newcommand{\asshare}[2]{\angled{#1}_{#2}}     
\newcommand{\asshare}[2]{\ensuremath{{#1}_{#2}}}                      % secret shared value by party #2
\newcommand{\pk}{\mathit{pk}}                                         % public key
\newcommand{\sk}{\mathit{sk}}                                         % private key
\newcommand{\paillier}[1]{\dbracked{#1}}                         % Paillier encrypted value
\newcommand{\distrs}[2]{\ensuremath{\mathcal{#1}_{#2}}}               % Random Distribution in the set #2
\newcommand{\distr}[1]{\ensuremath{\mathcal{#1}}}                     % Random Distribution
\newcommand{\unif}[1]{\set{U}_{#1}}                                   % Uniform random distribution in the set #1
\newcommand{\sample}[2]{\ensuremath{{#1}\oset{\supertiny{ \$}}{\leftarrow}{#2}}}% Sample value #1 from distribution #2
\newcommand{\sampleAppendT}[2]{\ensuremath{{#1}\uset{\centerdot}{\oset{\supertiny{ \$}}{\leftarrow}}{#2}}}% Assign value #2 to variable #1, and append variable #1 to transcript
\newcommand{\assignAppendT}[2]{\ensuremath{{#1}\uset{\centerdot}{\leftarrow}{#2}}}% Assign value #2 to variable #1, and append variable #1 to transcript
\newcommand{\appendT}[1]{\ensuremath{({#1})_{\centerdot}}}
\newcommand{\checkif}[3]{\ensuremath{{#1}\oset{\supertiny{ ?}}{#2}{#3}}}  % Check condition
\newcommand{\isequal}[2]{\checkif{#1}{=}{#2}}                         % Check equality
\newcommand{\isgeq}[2]{\checkif{#1}{\geq}{#2}}                        % Check greater or equal
\newcommand{\isgt}[2]{\checkif{#1}{<}{#2}}                            % Check greater
\newcommand{\poly}[1]{\ensuremath{\bm{\mathrm{#1}}}}                  % Polynomial
\newcommand{\polyx}[2]{\ensuremath{\poly{#1}({#2})}}           % Polynomial evaluated on point #2
\newcommand{\coeff}[2]{\ensuremath{\poly{#1}_{#2}}}            % Polynomial coefficient #2
\newcommand{\assign}[2]{\ensuremath{{#1}\leftarrow{#2}}}              % Assign value #2 to variable #1
\newcommand{\concat}{\ensuremath{\;||\;}}                             % Concatenation operator
\newcommand{\equals}{\ensuremath{\!=\!}}                              % Equivalence operator
\newcommand{\ins}{\ensuremath{\!\in\!}}                               % In operator
\newcommand{\inss}{\ensuremath{\!\!\in\!\!}}                          % In operator
\newcommand{\by}{\ensuremath{\!\times\!}}                             % Dimension multiplication (e.g., 3 by 2 matrix)
\newcommand{\str}[1]{\scalebox{0.75}{\ensuremath{``\mathtt{#1}"}}}      % String
\newcommand{\setminuss}{\ensuremath{\!\setminus\!}}                   % Set minus

    \newcommand{\specversion}{Version 0.9}

%-----------------------------------TITLE--------------------------------------%
    \title{Cryptographic Specification of Bron.xyz\\(\specversion)}
    \author{Alireza Rafiei, Alberto Ibarrondo, Mateusz Kramarczyk \\ Hoang Ong}
    \date{\today}

\begin{document}

%-------------------------------TITLE PAGE & ToC-------------------------------%
\maketitle
\begin{abstract}
    This document contains the specification of the cryptographic protocols and primitives that serve as base for the Multi-Party Computation (MPC) platform of \href{https://bron.xyz}{Bron}. This specification serves a reference for the implementation of all the techniques described in it.

    As a custody solutions provider, Bron's main objective with its MPC platform is to provide state-of-the-art protection for digital assets (e.g., cryptocurrencies) while allowing its clients to operate with them with the utmost security guarantees. To this end, we target distributed variants of Elliptic-Curve (EC) signature schemes commonly used to sign transactions in popular blockchains (ECDSA, Schnorr/EdDSA, BLS). We select MPC protocols that are proven secure against malicious adversaries in a $t$-out-of-$n$ threshold setting: $n$ parties generate and hold shares of the underlying private key, and any subset of $t$ parties can sign a transaction. This threat model tolerates up to $t-1$ static corruptions throughout the protocol execution. Our chosen distributed signing protocols are:
    \begin{itemize}
        \item DKLs23~\cite{DKLS23} and Lindell17~\cite{lindell17}\footnote{The latter only supports $t=2$, for any $n$.} for threshold ECDSA.
        \item Lindell22~\cite{L22} for threshold Schnorr.
    \end{itemize}
    We also specify the cryptographic primitives required by these protocols, including Distributed Key Generation (DKG), Oblivious Transfer (OT), proofs of knowledge (PoK), Cryptographically-Secure Pseudo-Random Number Generation (CSPRNG) and other general primitives.
\end{abstract}
\newpage
\tableofcontents
\clearpage

%----------------------------------SECTIONS------------------------------------%


%------------------------------------------------------------------------------%
\section{Introduction}\label{sec:introduction}
%------------------------------------------------------------------------------%

%Our goals / implementation points/ misc. [Alireza]
% • UC Secure, in real world (which is how we’ll rule out FROST, to be discussed later)
%     □ We don’t care about UC in DKG. (We might have to have a bad RSA keygen later)
% • Well-tested, minimal cryptography assumptions if we can help it (with BLS, we can’t)
% • Reliance on peer review instead of coming up with something new (because of high stakes, money etc)
% • Should be semi noninteractive or fully non interactive.
% • Why we don’t care about constant time.
% • Number of parties are small and there is CA (which why we don’t to NTT or similar)
% Various devices to be run


%------------------------------------------------------------------------------%
\subsection{Our Goals}\label{sec:our_goals}
%------------------------------------------------------------------------------%

This document describes in detail the suite of cryptographic protocols supported by the MPC-based signing services of \href{https://bron.xyz}{Bron}.


As a custody solutions provider, \href{https://bron.xyz}{Bron} aims to provide state-of-the-art protection for digital assets (e.g., cryptocurrencies) while allowing its clients to operate with them with the utmost security guarantees. To this end, we carefully select and implement several distributed variants of Elliptic-Curve (EC) signature schemes commonly used to sign transactions in popular blockchains (ECDSA, Schnorr/EdDSA, BLS), alongside their required primitives.
We focus on $t$-out-of-$n$ signing protocols where $n$ participants jointly generate shares of a secret key (via a \textit{Distributed Key Generation} protocol, or DKG), and $t$ out of these $n$ participants can jointly sign a public message under the distributed secret key (\textit{threshold signing}).
Our real-world use cases lead us to favor protocols tailored for small groups of participants (e.g., $n < 100$), with special interest in the $t=2$ case\footnote{Various optimizations could be made to increase performance for $n \gg 100$.}.

We conducted exhaustive literature reviews to choose which protocols to support, picking protocols whose security is guaranteed under minimal and/or well-tested and well-understood falsifiable assumptions. As our main threat model, the selected DKG \& signing protocols are proven secure against a malicious adversary statically corrupting up to $t-1$ parties, allowing the remaining honest parties to detect misbehavior and abort upon detection. As a requirement for some of our chosen protocols, this abort mechanism must be \textit{global}: all concurrent executions of that same protocol must stop upon detection of misbehavior in any of them. Whenever possible, we favor \textit{identifiable} abort mechanisms, where the cheating party/ies can be identified.

Casting aside various restrictions on the context in which these protocols are run, we favor protocols that were proven secure under concurrent composition in the Universal Composability (UC) paradigm~\cite{canetti2001universally}. Given that many instances of the signing protocols will run concurrently, we restrict our threshold signing choices to UC-secure-only protocols, guaranteeing that their security remains independent if any other runs of the same protocol takes place at the same time. \footnote{We could relax this requirement for the DKG protocols, as they are expected to run only once per key.}.

While we make a conscious effort to avoid non Constant-Time (CT) implementations of protocols, we do not seek nor claim complete CT implementations. We argue that a local timing attack in one of the participants of a protocol is strictly weaker than a malicious corruption, already covered by our threat model\footnote{Furthermore, we obviate remote timing attacks due to the inherently volatile nature of communications (latency, jitter), expecting our best-effort CT approaches to shield in these seemingly impossible cases}.

Because the stakes are high, and in the absence of standardization, we opt for influential peer-reviewed protocols/primitives backed by ample community convergence. Following our security principles and the financially sensitive contexts in which these protocols are being used, we stress that we have not developed any new primitives. Instead, this document aims to unambiguously specify these protocols as they are to be implemented. In the limited instances where a customization to the protocol is necessary, such customizations are carefully analyzed and meticulously presented in this document, often following an explicit confirmation by the authors of the protocols themselves. Lastly, we submitted both this document and our implementation to independent domain-expert auditing entities\footnote{\textit{TrailOfBits} conducted an audit on the code and V0.9 of this document.}, incorporating their feedback and addressing any issues raised.


% ○ Scope:  [Alberto]
% • Mention this is just primitives. Networking layer, and a lot of application related stuff is not here. (this is where we say ledger will live etc)
% Link to each part.

%------------------------------------------------------------------------------%
\subsection{Scope}\label{sec:scope}
%------------------------------------------------------------------------------%

The present specification is limited scope to cryptographic protocols and their required primitives only. Therefore, we purposely leave out the details of networking aspects of these multi-party protocols, e.g., how to establish and maintain the communication channels among participants, as well as application-related implementation choices, e.g., using a ledger and/or a coordinator to synchronize a participant's status across multiple devices. 

The specification is organized as follows. We kick off with the preliminaries, describing the primitives required by our main protocols in \autoref{sec:preliminaries}. We introduce the single-party signing schemes in \autoref{sec:signatures}, generalizing them to multiple parties in \autoref{sec:threshold_signatures} and building up the complete protocols for threshold ECDSA in \autoref{sec:tecdsa}, for threshold Schnorr (closely related to EdDSA) in \autoref{sec:tschnorr} and for threshold BLS in \autoref{sec:tbls}.
\clearpage

%------------------------------------------------------------------------------%
\section{Preliminaries}\label{sec:preliminaries}
%------------------------------------------------------------------------------%

%------------------------------------------------------------------------------%
\input{src/sections/2_preliminaries/notation}

%------------------------------------------------------------------------------%
\subsection{Generic Cryptography}\label{sec:genericcryptography}

% \input{src/sections/2_preliminaries/2_2_security}
\inputSection{hashing}{hashing.tex}
\inputSection{hashing/tmmohash}{tmmohash.tex}
\inputSection{csprng}{csprng.tex}

\inputSection{commitments}{commitments.tex}

\inputSection{indcpa/paillier}{paillier.tex}

\inputSection{transcripts}{transcripts.tex}

%------------------------------------------------------------------------------%
\subsection{Proofs of Knowledge}\label{sec:proofs}

\inputSection{proofs}{proofs.tex}
\inputSection{proofs/sigma}{sigma.tex}
\inputSection{proofs/dlog}{dlog.tex}
\inputSection{proofs/dleq}{dleq.tex}
\inputSection{proofs/paillier}{paillier_proofs.tex}

%------------------------------------------------------------------------------%
\newpage

\subsection{Multi-Party Computation (MPC)}\label{sec:mpc}

\inputSection{threshold/sharing}{sharing.tex}
\inputSection{threshold/agreeonrandom}{agreeonrandom.tex}
\inputSection{threshold/sharing/zero}{zero.tex}
\inputSection{ot}{ot.tex}
% ---------------------------------------------------------------------------- %
\subsubsection{Our Multi-Party Computation Scenario}\label{sec:mpc_scenario}
% ---------------------------------------------------------------------------- %

We detail in this subsection all the assumptions made for the scenario in which our protocols are run. These are common in most of the MPC-based literature, and assumed as such in all the threshold signing protocols that we implement. We leave the concrete instantiation of these requirement out of this specification.

\paragraph{On setting the participants to a protocol.} All the protocols assume its set of participants to be formed prior to the execution of the protocol, by using some out-of-band mechanism. We name this as \textit{controller set}. Furthermore, in many of these signing protocols, a participant's ID is the point at which the Shamir polynomial is evaluated to output the share for that participant. As a result, each participant will know the participant IDs of all other participants, among other things, before engaging in any of the signing protocols.

We do not make any assumption on the specifics of the controller set formation mechanism. We will leave this issue to the MPC platform and the particular Bron solution using this primitive. However, more often than not, these mechanisms require some identifier to be assigned to the participants, which has nothing to do with the primitive itself. We consider this and will require each participant to maintain a mapping from their protocol participant ID to their controller set ID.

It is possible to remove the need to maintain such a mapping by evaluating the Shamir polynomial at each participant's controller set ID. This makes each party's internal state simpler to represent, at the cost of modifying protocol-specific data if a participant's controller ID changes. We think such a cost is too high. Nevertheless, suppose a change is to be made to represent participant IDs via their controller set ID. In that case, special care should be given so that they don't reduce to zero modulo $q$, forcing the Shamir polynomial to reveal the secret.

\paragraph{On participant roles.} We define three specific roles for our distributed signing protocols:
\begin{enumerate}
    \item A semi-trusted \textit{signature aggregator} role, denoted as $\SA$, serving the following responsibilities:
    \begin{enumerate}
        \item \label{sa:2} $\SA$ validates each partial signature and reports the misbehaving participant if it fails.
        \item \label{sa:3} $\SA$ aggregates the partial signatures received from the participating parties.
    \end{enumerate}

    This role may be assumed by any one of the parties in the controller set or by an external entity, provided that they know the participant's partial public key \footnote{A \textit{partial public key} of a participant with the secret key share of $\sk_i$ is defined to be $\pk_i \equiv \sk_i \cdot G$ where $G$ is the generator of the curve.}. Such a role allows for less communication overhead between signers and is often practical in a real-world setting~\cite{KG20}. $\SA$ is required, and whoever assumes this role will have access to the final signature.

    Note that $\SA$ is semi-trusted: A malicious $\SA$ can perform denial-of-service attacks and report misbehavior by participants falsely, but it cannot learn the private key or cause improper messages to be signed. If all parties assume the $\SA$ role, it is impossible to prevent the last party from withholding their partial signature. This is natural since we're operating in the no-honest majority without fairness or guaranteed output delivery \cite{CY14}. All parties assume $\SA$ is equivalent to the scenario without $\SA$ described.

    \item A \textit{Presignature Composer} role denoted by $\PC$ in charge of composing the right pre-signature for the participating parties from the preprocessed material. The existence of a $\PC$ implies non-interactivity of the signing round, and it is completely optional.

    \item A \textit{Signing Requester} denoted as $\SR$. $\SR$ is the entity who sends the plaintext $m$ to the parties and asks them for a signature \footnote{This role is equivalent to the role \textit{leader} in \cite{BCKMTZ22}}.
\end{enumerate}

It is \textbf{required} that all parties agree on who assumes $\PC$ and $\SA$ and $\SR$ at the beginning of the signing session since they need to know what message they should trust to sign, whether to generate pre-signature tuples or to whom they should send their partial signatures.

For simplicity, we assume $\PC$ to have at most one entity, and $\SR$ as well as $\SA$ to have exactly one entity. If it is desired for these sets to contain more entities, an agreement protocol should be run on their decision. The details of acceptable agreements protocols are highly context-dependent, and depending on the fault model assumed by the users of these primitives good choices will be different \footnote{For example, a client could select fixed signature aggregators outside of the controller set, and assume crash-fault}.

\paragraph{On Signing scenarios.} Various Bron use cases will require different selections of each role. Below, we point out how various modes of Bron's SASS offering can be realized:
\begin{itemize}
    \item \textbf{Cold Signing:} non-interactive signing, with $\PC=\SR=\{\party{client}\}$ and $\SA=\{\party{bron}\}$.
    \item \textbf{Hot Signing:} This is interactive signing where $\PC=\{\}$ and $\SR=\{\party{client}\}$ and $\SA=\{\party{bron}\}$
    \item \textbf{Proxy:} This is equivalent to \textbf{Hot Signing}.
\end{itemize}

\inputSection{network}{network.tex}


\clearpage

% a. General: [Alireza]
% § Mention the verify function and security game etc
\section{Digital Signatures}\label{sec:signatures}


A digital signature is an asymmetric cryptographic primitive for verifying the authenticity of digital messages or documents \cite{goldwasser1983strong,goldwasser1988digital,bellare1988sign,rompel1990one,goldwasser2008lecturenotes}. A valid signature on a message gives a recipient trust in that the message was authored by a sender known to the recipient. Digital signatures are commonly used for financial transactions, software distribution, and to detect forgery or tampering in communication.

Assuming that the private key remains secret, digital signatures are designed to be inherently \textit{resistant to forging}, making it computationally infeasible to generate a valid signature on behalf of a party without knowing that party's private key. They may also provide \textit{non-repudiation}, thus signer cannot successfully claim they did not sign a message. Digitally signed messages may be anything that can be represented as a bit-string, e.g., cryptocurrency transactions, electronic mail, or messages sent as part of cryptographic protocols.

We generalize digital signature schemes in Functionality \ref{alg:f_signature}:

\begin{functionality}
    \caption{$\quad\idealfndef{Signature}$}
    \label{alg:f_signature}
    \begin{algorithmic}[1]
        \begin{alginfo}
            Signature scheme, composed of three algorithms:
            \begin{itemize}[leftmargin=12pt]
                \item $\fndef[f_signature]{KeyGen}()\!\rightarrow\! (\sk,\pk)$, generates a private \& public key pair $(\sk, \pk)$
                \item $\fndef[f_signature]{Sign}(\sk, m)\!\rightarrow\! \sigma$ yields signature $\sigma$ for message $m$ \& private key $\sk$
                \item $\fndef[f_signature]{Verify}(\pk, m, \sigma)\!\rightarrow\! b \ins \{0,1\}$ verifies signature $\sigma$ for message $m$ and public key $\pk$, either accepting ($b\equals1$) or rejecting ($b\equals0$) it\footnotemark.
            \end{itemize}
        \end{alginfo}
    \end{algorithmic}
\end{functionality} 

\footnotetext{Equivalently, it returns \textit{valid} ($b\equals1$) for a valid signature and $\ABORT$ otherwise.}

The security of digital signature schemes is filled with nuances. In general, the security of a digital signature scheme is defined by a security game between a challenger and an adversary. Based on the capabilities granted to the adversary, we distinguish three basic attacks (more details in \cite{goldwasser1983strong}):
\begin{itemize}
    \item \textit{Key-Only Attack (KOA)}: the adversary knows only the public key of the signer, thus it can verify signatures of messages given to him.
    \item \textit{Known Signature Attack (KSA)}: the adversary is given the public key of the signer and message/signature pairs from a legal signer.
    \item \textit{Chosen Message Attack (CMA)}: The adversary is also given access to a signing oracle, a black-box that signs any message of the adversary's choice. Out of the three, this is the most powerful adversary and is considered the only realistic notion for many real-world scenarios.
\end{itemize}

\noindent The adversary's goal in the security game may be:
\begin{itemize}
    \item \textit{Existential Forgery (EF)}: succeed in forging the signature of one message, not necessarily of his choice.
    \item \textit{Selective Forgery (SF)}: succeed in forging the signature of some message of his choice.
    \item \textit{Universal Forgery (UF)}: succeeds in forging the signature of any message.
    \item \textit{Total Break}: succeed in computing the signer's secret key.
\end{itemize}

The signature scheme is then considered secure if the adversary's probability of winning the game for a given goal and attack is negligible. We refer the avid reader to \cite{goldwasser2008lecturenotes} for more formal definitions.

There exist many different digital signature schemes, each with their own performance characteristics. In this section, we focus our attention to the most common signature schemes used in blockchain protocols: ECDSA~\cite{ecdsa_rfc6979}, Schnorr / EdDSA~\cite{eddsa_rfc8032}, and BLS~\cite{irtf_bls_signature_05}. These three schemes are proven secure under the \textit{UF-CMA} security definition\footnote{Some signature schemes such as Schnorr / EdDSA go even beyond, as they constitute \textit{zero knowledge proofs of knowledge} of the secret key.}, the strongest combination out of the classical security notions presented above. All these schemes are based on elliptic curve cryptography, for specifically chosen curves.

\input{src/sections/3_signatures/3_ecdsa}
\input{src/sections/3_signatures/3_schnorr}
% d. BLS:   [Alireza]
% § Mention target curve for each algorithm
% § Signing modes
% Briefly mention use-cases
% ---------------------------------------------------------------------------- %
\subsection{BLS}\label{sec:bls}
% ---------------------------------------------------------------------------- %


The Boneh-Lynn-Shacham (BLS) signature scheme~\cite{boneh2001short} is a deterministic, non-malleable, and efficient signature scheme grounded on pairing-based cryptography, resorting to a type III bilinear pairing $\fn{e}\!: \G_1 \times \G_2 \rightarrow \G_T$ for signature verification. Subject of an RFC draft~\cite{irtf_bls_signature_05}, the BLS signature scheme is provably secure (the scheme is \textit{existentially unforgeable} under \textit{adaptive chosen-message attacks}) in the random oracle model assuming hardness of a variant of the computational Diffie-Hellman problem. $\G_1$ and $\G_2$ are elliptic curve groups with the same characteristic $q$ defined over fields of order $p$ and $p^2$ respectively, and $\G_T$ is a target group over a field of order $p^{12}$.


\begin{scheme}[ht]
    \caption{$\quad\fndef{BLS}$}
    \label{alg:bls}
    \begin{algorithmic}[1]
        \begin{alginfo}
            The pairing-based signing scheme from \cite{boneh2001short} over curve BLS 123 81 \cite{irtf_bls_signature_05}, instantiated with short keys \textit{wlog} and with all rogue key prevention schemes. It uses hash functions $\fn{H}_{\G_1}$ and $\fn{H}_{\G_2}$ over $\G_1$ and $\G_2$ respectively. $\mathbf{BLS123\mhyphen81}.\fn{Verify}(m, \sigma)$ is a pairing-based verification function returning $\isequal{(\fn{H}_{\G_1}(m) \times \g{Y}^{-1}) \cdot (\g{G}_1 \times \sigma)}{1}$.
        \end{alginfo}
        \Require {A unique session identifier $sid$, a message $\vect{m}$ to be signed, a public key $\g{Y}_i$, and a private key $x_i$ for each signer $\party{i} \;\forall i \in \range{t}$.}

        \Ensure {A partial signature $\sigma_i$ for each player $\party{i} \;\forall i \in \range{t}$ after round 1, and a signature $\sigma$ after aggregation, }

        \vspace{-12pt}
        \setalglineno{1}
        \AlgFnctZero{\fndef[bls]{Sign}}{i}{$\vect{m}$}{$\sigma_i$}
            \State If $\fn{MessageAug}$, $\assign{\vect{m}}{\g{Y}_i \concat  \vect{m} }$.
            \State If $\fn{PoP}$,  $\assign{\pi_i}{\fn x_i  \times \fn{H}_{\G_2}(\g{Y}_i) }$.
            
            \State $\sigma_i \gets x_i \times \fn{H}_{\G_2}(\vect{m})$
            \State \Return $\sigma_i$ as the partial signature, attaching $\pi_i$ to it if $\fn{PoP}$.
      
        \vspace{-12pt}
        \setalglineno{1}
        \AlgFnct{\fndef[bls]{Verify}}{}{$\{\sigma_i\}_{i \in \range{t}}$}{\textit{valid}}
            \For{$i \in \range{t}$}
                \State If $\fn{Basic}$, ensure all $\vecti{m}{i}$ are unique. $\ABORT$ otherwise.
                \State If $\fn{PoP}$, check if $\fn{\mathbf{BLS123\mhyphen81}.Verify}(\g{Y}_i, \pi_i)$ is \textit{valid}. $\ABORT$ otherwise.
                \State If $\fn{MessageAug}$, $\assign{\vect{m}}{\g{Y}_i \concat  \vect{m} }$.
                \State Check if $\fn{\mathbf{BLS123\mhyphen81}.Verify}(\g{Y}_i, \sigma_i)$ is \textit{valid}. $\ABORT$ otherwise.
            \EndFor
            \Statex \Return \textit{valid}.

    %TODO: Batch aggregate verify

    \end{algorithmic}
\end{scheme}


Based on the choice of what group to use for what scheme element, BLS supports two modes of operation:
\begin{itemize}
    \item \textbf{Short public keys, long signatures}: Signatures are longer and slower to create, verify, and aggregate but public keys are small and fast to aggregate. Used when signing and verification operations not computed as often or for minimizing storage or bandwidth for public keys. $\G_1$ is used as the key group, and $\G_2$ as the signature group. 
    \item \textbf{Short signatures, long public keys}: Signatures are short and fast to create, verify, and aggregate but public keys are bigger and slower to aggregate. Used when signing and verification operations are computed often or for minimizing storage or bandwidth for signatures. $\G_2$ is used as the key group, and $\G_1$ as the signature group.
\end{itemize}

The BLS signature can be composed into a multi-signature scheme with batch verification. To do so, an additional mechanism must be introduced to prevent rogue public-key attacks (i.e., a malicious party can carefully craft a signature that negates the contributions of other parties and reveals a secret key). The BLS signature scheme supports three rogue key prevention schemes: 

\begin{enumerate}
    \item \textbf{Basic}: Requiring all messages to be unique.
    \item \textbf{Message Augmentation}: Prepends the public key of the signer to the message thereby making each message unique.
    \item \textbf{Proof of Possession (PoP)}: Every signature is accompanied by a signature of the public key acts as proof of possession of its secret key.
\end{enumerate}

We diverge slightly from \cite{irtf_bls_signature_05} by employing an optimization trick on the final exponentiation\footnote{From \url{https://hackmd.io/benjaminion/bls12-381\#Final-exponentiation}}. Crucially, we also include the message hashing as part of the verification process in order to avoid some attacks. %TODO: add reference
\clearpage

%------------------------------------------------------------------------------%
\section{Threshold Signatures}\label{sec:threshold_signatures}
%------------------------------------------------------------------------------%

\inputSection{threshold/dkg}{dkg.tex}
\input{src/sections/4_threshold_signatures/interactive}
% c. Bron Non interactive  [Alireza]
% § Overall mention n-1 rounds in parallel
% - Accepts an index to the presignature batch. What presignatures are, what presignature bach are, check that indeces are not reused.
% § Define Pregen, Presignature composer, and Signature aggregator
% § Talk about ledger, and certification process (everyone signing their own contribution to the presignatures).
% § Mention no additive key derivation so no need for rerandomization
% Communicate that this is outside of the primitive scope.
%------------------------------------------------------------------------------%
\subsection{Non-Interactive signing}\label{sec:noninteractive_signing}
%------------------------------------------------------------------------------%
In contrast to the above, the non-interactive signing mode involves running the message-independent rounds (typically $k-1$ out of the $k$ round threshold signing protocols) ahead of time, a process that we refer to as \textit{Pregen}. The resulting material, the \textit{pre-signatures}, is stored in shares for later use by the parties that created them. This ahead-of-time preprocessing can be performed with multiple instances of the protocol running in parallel to obtain a batch of pre-signatures. To tie the generated pre-signatures to their generated parties, each pre-signature share is signed (a.k.a. \textit{certified}) by its generating party.

Upon receiving a request to sign a certain message/transaction, the pre-signatures are used for signing this message in a non-interactive fashion, by having the same subset of $t$ signing parties execute the final message-dependent round(s) of the protocol. This typically involves some local computation and one round of communication to send the signature shares to the \textit{Signature Aggregator} for it to combine them into a full signature.

The non-interactive signing mode is particularly useful in scenarios where the signing parties are not all online at the time of signing, or as a way to speed-up signing operations by leveraging periods of time when the signing servers have low load (e.g., at night).

This mode incorporates an additional role, the \textbf{pre-signature composer} $\PC$ defined in \autoref{sec:mpc_scenario}. If participants do not hold their own pre-signatures, $\PC$ will select the right pre-signature with the right certificate. In a common configuration, the $\SR$ will also act as $\PC$ by picking the index of the pre-signature to be used for a non-interactive signing request.


In this mode, special care must be taken to ensure that the pre-signatures are not reused. Indeed, if a pre-signature is used to sign more than one message, the security of the protocol is compromised. To tackle this issue, we employ indexing inside the batch of pre-signatures in order to choose the pre-signature to be used in a non-interactive signing request. Handling the choice of index for each request while ensuring that this index wasn't previously used is outside of the scope of this document. Nonetheless, we stress the need for some kind of synchronization mechanism between the parties to avoid batch index reuse. This mechanism could take the form of a ledger (e.g., writing information to a blockchain) that parties should read upon each message signing request, or as an additional coordination protocol between the parties.

\clearpage

%------------------------------------------------------------------------------%
\newpage
\section{Threshold ECDSA}\label{sec:tecdsa}
%------------------------------------------------------------------------------%
 
\inputSection{threshold/tsignatures/tecdsa}{tecdsa.tex}
\clearpage

%------------------------------------------------------------------------------%
\section{Threshold Schnorr}\label{sec:tschnorr}
%------------------------------------------------------------------------------%

\inputSection{threshold/tsignatures/tschnorr}{tschnorr.tex}
\clearpage

%------------------------------------------------------------------------------%
\section{Threshold BLS}\label{sec:tbls}
%------------------------------------------------------------------------------%

\inputSection{threshold/tsignatures/tbls}{tbls.tex}
\clearpage

\bibliographystyle{alpha}
\bibliography{src/bib/tecdsa.bib,src/bib/refresh.bib,src/bib/primitives.bib,src/bib/tschnorr.bib,src/bib/various.bib,src/bib/tbls.bib}

\appendix
\section{Appendix}\label{sec:appendix}

\input{src/sections/appendix/A1_twothirds.tex}

\input{src/sections/appendix/A2_curves.tex}

\end{document}
