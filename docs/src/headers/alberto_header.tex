\usepackage{graphicx}

\usepackage{tikz}
\usepackage[utf8]{inputenc}
\usepackage[english]{babel}
% \usepackage{amssymb}
\usepackage{amsmath}
\usepackage{booktabs}
\usepackage{graphicx} % http://ctan.org/pkg/graphicx
\usepackage{url}
\usepackage{bm}       % bold greek letters
\usepackage{algorithm} 
\usepackage[noend]{algorithmic}[1]
\usepackage{tcolorbox}
\usepackage{balance}
\usepackage{diagbox} % Diagonal split of table cells
\usepackage{multirow} % Multirow table cells
\usepackage{enumitem}
\usepackage{float}
\usepackage{footnote}
\usepackage{footmisc}
\usepackage{tablefootnote} % Footnotes in tables with \tablefootnote
\usepackage{mathtools}  % \mathclap to compress formula
\RequirePackage{fix-cm}

% \renewcommand{\thefootnote}{\fnsymbol{footnote}} % Footnote symbols

% Circled numbers, command \circled{1}
\usepackage{tikz}
\newcommand*\circled[1]{\tikz[baseline=(char.base)]{
            \node[shape=circle,draw,inner sep=1pt] (char) {#1};}}


%% custom paragraph that stands out more
\newcommand{\fakeparagraph}[1]{\vskip 0pt\noindent\textbf{#1 }}


\newcommand{\party}[1]{\textsf{P}_{#1}}
\newcommand{\role}[1]{\textsf{R}_{#1}}

% Math operators
\makeatletter
\newsavebox{\@brx}
\newcommand{\llangle}[1][]{\savebox{\@brx}{\(\m@th{#1\langle}\)}%
  \mathopen{\copy\@brx\mkern2mu\kern-0.9\wd\@brx\usebox{\@brx}}}
\newcommand{\rrangle}[1][]{\savebox{\@brx}{\(\m@th{#1\rangle}\)}%
  \mathclose{\copy\@brx\mkern2mu\kern-0.9\wd\@brx\usebox{\@brx}}}
\makeatother

\newcommand{\encr}[1]{\left \langle {#1} \right \rangle_{\mathbf{pub}}} 
\newcommand{\angled}[1]{\left \langle {#1} \right \rangle} 
\newcommand{\dangled}[1]{\llangle {#1} \rrangle} 
\newcommand{\bracked}[1]{\left[ {#1} \right]} 
\newcommand{\dbracked}[1]{ \llbracket {#1} \rrbracket} 

\newcommand{\bxor}[0]{\! \oplus \!}
\newcommand{\band}[0]{\! \wedge \!}
\newcommand{\toint}[1]{\ensuremath{\lfloor#1\rceil_{\mathbb{Z}_{2^n}}}}

%%> ROTATING COLUMNS: https://tex.stackexchange.com/questions/32683/
\usepackage{xparse}   % http://ctan.org/pkg/xparse
% Rotation: \rot[<angle>][<width>]{<stuff>}
\NewDocumentCommand{\rot}{O{45} O{1em} m}{\makebox[#2][l]{\rotatebox{#1}{#3}}}%

% Notes, Missing references and TODOs				
\newcommand{\notte}[1]{\color{red}(#1!)\color{black}}
\newcommand{\missref}[1][]{\notte{[REF#1]}}
\newcommand{\todo}[1]{\textcolor{orange}{[Todo: #1]}}

% Math Environments
\renewcommand{\algorithmicrequire}{\textbf{Players:}}
\renewcommand{\algorithmicensure}{\textbf{Output:}}
\newcommand{\PARAMS}{\hspace{-0.55cm}\textbf{Input:}\space\space}

\newcounter{protocol}
\newenvironment{protocol}[1][htb]{%
  \floatname{algorithm}{Protocol}% Update algorithm name
  \begin{algorithm}[#1]%
}{\end{algorithm}}

\newcounter{algo}
\newenvironment{algo}[1][htb]{%
  \renewcommand{\algorithmicrequire}{\textbf{Inputs:}}
  \renewcommand{\algorithmicensure}{\textbf{Output:}}
  \floatname{algorithm}{Algorithm}% Update algorithm name
  \begin{algorithm}[#1]%
}{\end{algorithm}}


\makeatletter
\newcounter{simulator}
\newenvironment{simulator}[1][htb]{%
  \let\c@algorithm\c@simulator
    \renewcommand{\ALG@name}{Simulator}% 
   \begin{algorithm}[#1]%
  }{\end{algorithm}}
\makeatother

% INDENTATOR
% Environment to indent right content based on first parameter, then indent the first line of the content by the second parameter.
% e.g., indenting the first line by 20pt and the rest by 30pt:
%      \begin{indent}{30pt}{-10pt} 
%           My Paragraph.
%      \end{indent}
\newenvironment{indentator}[2]
  {%
    \setlength{\leftskip}{#1}
    \bgroup
    \setlength{\parindent}{#2}
  }
  {
    \egroup
    \setlength{\leftskip}{0pt}
  }


% Simulator for real-ideal world proof

\newtheorem{theorem}{Theorem}
\newtheorem{definition}{Definition}


\newcommand\blfootnote[1]{%
  \begingroup
  \renewcommand\thefootnote{}\footnote{#1}%
  \addtocounter{footnote}{-1}%
  \endgroup
}