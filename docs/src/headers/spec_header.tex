\usepackage[utf8]{inputenc} 
\usepackage[section]{algorithm}% Float environment for algorithms
\usepackage{algorithmicx}       % Syntax for algorithms
\usepackage[noEnd=true,indLines=false,beginComment=,rightComments=true]{algpseudocodex} % Customized algorithmicx
\usepackage{lineno}             % line numbers
\usepackage{bm}                 % bold math
\usepackage[section]{placeins}
\usepackage{xcolor}
\usepackage{comment}
\usepackage{amsmath}
\usepackage{import}             % For \subimport
\usepackage{amssymb}
\usepackage{enumitem}           % [leftmargin=12pt] in itemize
\usepackage{graphicx}           % scalebox
\usepackage{tcolorbox}          % Colored boxes for ideal functionalities
\usepackage{booktabs}           % For \toprule, \midrule and \bottomrule
\usepackage[only,llbracket,rrbracket]{stmaryrd} % For \llbracket and \rrbracket

\usepackage{amsthm}             % Theorems
\newtheorem{theorem}{Theorem}
\newtheorem{case}{Case}

\usepackage{amsfonts}           % Short minus
\DeclareMathSymbol{\shortminus}{\mathbin}{AMSa}{"39}

\usepackage[hypertexnames=false]{hyperref}           % Colour links
\hypersetup{
  colorlinks=true,
  linkcolor=blue, %Colour of internal links
}

\newcommand{\hyperrefblk}[2]{\hyperref[#1]{\color{black}{#2}}} % Black hyperlinks


%------------------------------------------------------------------------------%
%-------------------------------FILE LOCATION----------------------------------%
%------------------------------------------------------------------------------%

\newcommand{\codepath}{../pkg}
\newcommand{\inputAlgorithm}[2]{\inputfrom{\codepath/#1/docs/algorithms}{#2}}
\newcommand{\inputFunctionality}[2]{\inputfrom{\codepath/#1/docs/functionalities}{#2}}
\newcommand{\inputSection}[2]{\inputfrom{\codepath/#1/docs/sections}{#2}}


%------------------------------------------------------------------------------%
%-------------------------------GENERAL SETUP----------------------------------%
%------------------------------------------------------------------------------%
% Autoref with capitalized names
\renewcommand{\figureautorefname}{Figure}
\renewcommand{\tableautorefname}{Table}
\renewcommand{\partautorefname}{Part}
\renewcommand{\appendixautorefname}{Appendix}
\renewcommand{\chapterautorefname}{Chapter}
\renewcommand{\sectionautorefname}{Section}
\renewcommand{\subsectionautorefname}{Section}
\renewcommand{\subsubsectionautorefname}{Section}


%------------------------------------------------------------------------------%
%--------------------------------ALGORITHM ENV---------------------------------%
%------------------------------------------------------------------------------%
% ALGORITHM INPUTS, OUTPUTS, PLAYERS
\renewcommand{\algorithmicrequire}{\textbf{Inputs:}}
\renewcommand{\algorithmicprocedure}{}
\renewcommand{\algorithmicensure}{\textbf{Outputs:}}
\algnewcommand{\Players}{\item[\textbf{Players:}]}
\algnewcommand{\Params}{\item[\textbf{Parameters:}]}

% DESCRIPTIONS and COMMENTS
\algnewcommand{\Note}{\item[\textbf{Note:}]}
\algnewcommand{\Descr}[1]{{\setlength{\itemindent}{-\leftmargin}\item[]{#1}}}
\algnewcommand{\Descrx}[1]{{\setlength{\itemindent}{-\labelwidth}\item[]\textit{#1}}}
\algnewcommand{\Commentg}[1]{\Comment{\textcolor{gray}{#1}}}

% PROTOCOL
%  with "Players", "Input" and "Output"
\newcounter{protocol}
\makeatletter
\newenvironment{protocol}[1][htb]{%
    \let\c@algorithm\c@protocol
    \renewcommand{\ALG@name}{Protocol}
    \begin{algorithm}[#1]%
    }{
    \end{algorithm}
}
\makeatother

\newcounter{scheme}
\makeatletter
\newenvironment{scheme}[1][htb]{%
    \let\c@algorithm\c@scheme
    \renewcommand{\ALG@name}{Scheme}
    \begin{algorithm}[#1]%
    }{
    \end{algorithm}
}
\makeatother

% FUNCTIONALITY
%  with "Players", "Input" and "Output"
\newcounter{functionality}
\makeatletter
\newenvironment{functionality}[1][htb]{%
    \let\c@algorithm\c@functionality 
    \renewcommand{\ALG@name}{Functionality}
    \begin{tcolorbox}[left=3pt,right=3pt]
    \vspace{-15pt}
        \begin{algorithm}[H]
        }{
        \end{algorithm}
    \end{tcolorbox}
    }
\makeatother
  
% Set Algorithmic line number
\newcommand{\setalglineno}[1]{
  \setcounter{ALG@line}{\numexpr#1-1}}

% Gray comment in protocols / algorithms
\newcommand{\ABORT}{{\textcolor{red}{\textsc{ABORT}}}}
\newcommand{\grayhfill}{\color{lightgray}\hrulefill}

% Change margin
\newenvironment{changemargin}[2]{%
  \begin{list}{}{%
    \setlength{\topsep}{0pt}%
    \setlength{\leftmargin}{#1}%
    \setlength{\rightmargin}{#2}%
    \setlength{\listparindent}{0pt}%
    \setlength{\labelsep}{0pt}%
    \setlength{\itemindent}{0pt}%
    \setlength{\parsep}{\parskip}%
  }%
  \item[]}{\end{list}}

% Algorithm info (paragraph at the beginning of the algorithm)
\newenvironment{alginfo}{
    \item[]
    \begin{changemargin}{-16pt}{0pt} 
} {\end{changemargin}\vspace{4pt} }

% Round syntax style
\algnewcommand{\RoundFmt}[2]{\textup{#1}$.$\textbf{\textsf{#2}}}
% First round in protocol
\algnewcommand{\AlgRoundZero}[4]{
    \Statex % Always use one empty state to mark the beginning of the protocol.
    \begin{changemargin}{-32pt}{0pt} 
        \Descrx {\RoundFmt{#1}{#2}}({#3})$\dashrightarrow${#4}
    \end{changemargin}}
% Next protocol rounds
\algnewcommand{\AlgRound}[4]{
    \begin{changemargin}{0pt}{0pt} 
        \Descrx {\RoundFmt{#1}{#2}}({#3})$\dashrightarrow${#4}
    \end{changemargin}}
\algnewcommand{\AlgRoundInput}{\par$\quad$} % \item for inputs of each round.
\algnewcommand{\AlgSeparator}{{\color{lightgray}\hrulefill}\vspace{4pt}} % Round separator.

% Function syntax style
\algnewcommand{\FunctFmt}[2]{\ensuremath{\bm{\textsf{#1}}_{#2}}}
% First Function in scheme
\algnewcommand{\AlgFnctZero}[4]{
    \Statex % Always use one empty state to mark the beginning of the protocol.
    \begin{changemargin}{-32pt}{0pt} 
        \Descrx {\FunctFmt{#1}{#2}}({#3})$\dashrightarrow${#4}
    \end{changemargin}}
% Next functions
\algnewcommand{\AlgFnct}[4]{
    \begin{changemargin}{0pt}{0pt} 
        \Descrx {\FunctFmt{#1}{#2}}({#3})$\dashrightarrow${#4}
    \end{changemargin}}

%------------------------------------------------------------------------------%
%---------------------------------CUSTOM ENVS----------------------------------%
%------------------------------------------------------------------------------%
% INDENTATION
% Environment to indent right content based on first parameter, then indent the first line of the content by the second parameter.
% e.g., indenting the first line by 20pt and the rest by 30pt:
%      \begin{indent}{30pt}{-10pt} 
%           My Paragraph.
%      \end{indent}
\newenvironment{indentator}[2]
  { \setlength{\leftskip}{#1}
    \bgroup
    \setlength{\parindent}{#2}
  }{\egroup
    \setlength{\leftskip}{0pt}
  }


%------------------------------------------------------------------------------%
%----------------------------------AUXILIARY-----------------------------------%
%------------------------------------------------------------------------------%

% TODOs
\newcommand{\todo}[1]{\textcolor{orange}{[Todo: #1]}}

% Super-tiny font
\newcommand{\supertiny}[1]{\scalebox{.5}{#1}}

% \oset macro for tight overset, from https://tex.stackexchange.com/a/194805/288479
\makeatletter
\newcommand{\oset}[3][0ex]{%
  \mathrel{\mathop{#3}\limits^{
    \vbox to#1{\kern-2\ex@
    \hbox{$\scriptstyle#2$}\vss}}}}
\makeatother

% \uset macro for tight underset, from https://tex.stackexchange.com/a/194805/288479
\makeatletter
\newcommand{\uset}[3][0ex]{%
  \mathrel{\mathop{#3}\limits_{
    \vbox to#1{\kern-10\ex@
    \hbox{$\scriptstyle#2$}\vss}}}}
\makeatother